% Software-Agenten im Internet
% Copyright Florian Pircher <https://addpixel.net/>

\RequirePackage[l2tabu,orthodox]{nag} % enforce best practices

\documentclass{article}

% Layout
\usepackage[a4paper]{geometry}
\usepackage{booktabs} % nicer tables
\usepackage[hang,flushmargin]{footmisc} % don't indent all the things
% Text
\usepackage[utf8]{inputenc}
\usepackage[english,ngerman]{babel}
\usepackage{hyperref}
\usepackage[style=authortitle-icomp]{biblatex}
% Typography
\usepackage{microtype} % tweak spacing and sizing so everything fits nicely
\usepackage[babel,german=guillemets]{csquotes} % harmonize (french) quotes

% Basic Configuration
\bibliography{master.bib}

\begin{document}
  \title{Software-Agenten im Internet}
\author{
  Florian Pircher\\
  Technologische Fachoberschule\\
  Oberschulzentrum Fallmerayer\\
  Brixen, Italien
}
\date{\today}
\maketitle
  \renewcommand{\abstractname}{Abstract}

\begin{abstract}
  Menschen sind schon lange nicht mehr die einzigen Nutzer des Internets.
  Beinahe 50\,\% aller Webseiten-Aufrufe werden von autonomer Software
  getätigt. Diese auch als Bots bezeichnete Softwares agieren in den Schatten
  des Netzes. Unbemerkt indexieren sie Webseiten, verbreiten Spam, legen
  gefälscht Profile an oder versuchen in Datenbanken einzubrechen.
  
  Hindernisse wie CAPTCHAs oder Honeypots galten bislang als effektive
  Gegenmittel, allerdings verhilft der rapide Fortschritt im Feld der
  Künstlichen Intelligenz modernen Bots auch derartige Barrieren zu
  durchbrechen. Im Bereich E-Mail-Spam findet ein unablässiger Kampf zwischen
  Spam-Bots und Klassifizierungsalgorithmen statt.
  
  Im ersten Teil dieser Arbeit wird das Verhalten und Vermögen von
  Software-Agenten untersucht. Schwerpunkte bilden dabei die Themen Spam und
  Künstliche Intelligenz. Der zweite Teil beschreibt die Anwendung des
  gewonnenen Wissens in Form der Entwicklung eines eigenen Bots, der autonom
  durch das World Wide Web navigiert.
\end{abstract}

  
  \tableofcontents
  
  \section{Einleitung} % (fold)
\label{sec:einleitung}
% section einleitung (end)
  
  \section{Überblick} % (fold)
  \label{sec:uberblick}
  \subsection{Software-Agenten} % (fold)
\label{sub:software_agenten}
Ein Software-Agent ist eine Computer-Applikation, welche autonom vordefinierte Ziele verfolgt. Alternative Namen umfassen: \emph{Agent}, \emph{autonome Software}, \emph{Softbot}, \emph{Bot} oder im Bereich der Künstlichen Intelligenz als \emph{KI} oder \emph{A.I.}
% subsection software_agenten (end)
  % section uberblick (end)
  
  \section{Beobachten und Verstehen} % (fold)
  \label{sec:beobachten_und_verstehen}
  \subsection{Spam} % (fold)
\label{sub:spam}
Unerwünschte Nachrichten im Internet, größtenteils Werbeangebote,
Phishing-Attacken oder Übermittler von Schadsoftware, werden als Spam
bezeichnet. Spam-Nachrichten stellen per definitionem eine leidige
Kommunikationsform dar. Die gewöhnlichste Art, namhaft durch ihre Omnipräsenz
im Leben mit dem Internet, stellt der E-Mail-Spam dar.

Der Begriff fand noch vor dem Erscheinen des \emph{World Wide Web} anklang,
beschrieb damals jedoch eine Flut an unerwünschten
Nachrichten.\footcite{originTermSpam} Erste Aufkommen des Phänomens wurden im
ARPANET\footcite{junkMailProblem} und im USENET beobachtet.

Der Name \emph{Spam} entspringt dem gleichnamigen Dosenfleisch, welches im
Zweiten Weltkrieg in großen Mengen an Soldaten verteilt
wurde.\footcite{lifeDuringSecondWW} Der Markenname Spam® (kurz für \emph{Spiced
Ham}) setzte sich auf diese Weise rasch im Vereinigten Königreich als Deonym
für Frühstücksfleisch durch. Die britische Comedy-Gruppe Monty Python griff im
Sketch \emph{Spam}\footcite{spamMontyPython} der BBC Serie \emph{Monty Python’s
Flying Circus} die Allgegenwärtigkeit des Fleischprodukts auf, weshalb dieses
bis heute als Symbol für einen unerwünschten Überschuss fungiert.

Mit dem Begriff \emph{Spam} werden heutzutage diverse Arten von Nachrichten
bezeichnet. Werbenachrichten sind das klassische Beispiel einer Spam-Nachricht.
Sie bieten zumeist dubiose Produkte oder Dienstleistungen an.

\subsubsection{E-Mail-Spam vermeiden}
\label{ssub:e-mail-spam-vermeiden}

Das Problem Werbung und andere unerwünschte Nachrichten zu erhalten, ist
nicht neu; Papier-Post belästigt Hausbesitzer seit Generationen.
Gegenmaßnahmen wie ›Bitte keine Werbung einwerfen‹-Kleber für das
Postfach sind etablierte Präventivmaßnahmen, die in einigen Staaten
gesetzlich verankert sind.{[}\^{}TK
\url{http://www.focus.de/finanzen/news/postwurfsendungen-unerwuenschte-reklame-im-briefkasten-ist-illegal_aid_699771.html}
2016-05-26{]} Digitale Spam-Nachrichten dagegen sind ein relativ neues
Phänomen und viele E-Mail-Nutzer fühlen sich durch eine Flut von
Spam-Nachrichten überlastet. Die folgenden Ratschläge empfehlen sich, um
E-Mail-Postfächer vor Spam zu schützen.

\paragraph{Spam-Filter}
\label{par:spam-filter}

Spam-Filter identifizieren und markieren Spam-Mails. Sie können direkt
auf einem Mail-Server oder in einem E-Mail-Client installiert werden.
Als Spam markierte Mails werden dabei in ein eigenes Postfach bewegt;
sollten wichtige E-Mails nicht im Posteingang erscheinen, kann man das
Spam-Postfach auf die erwarteten Nachrichten untersuchen. Viele
E-Mail-Clients erlauben das manuelle Markieren von Spam-Mails, sollte
eine E-Mail den Spam-Filter überlisten. Diese Aktion ist dem einfachen
Löschen der Nachricht insofern überlegen, als dass der Client die
markierte E-Mail auf Spam-Indikatoren hin untersucht um diese in seinen
Spam-Filter-Algorithmus zu integrieren.

\paragraph{Bilder unterbinden}
\label{par:bilder-unterbinden}

Eine drastische, wenn auch effiziente Maßnahme ist das vollständige
Blockieren aller Bilder. Dadurch, dass Bilder erst beim Darstellen einer
E-Mail geladen werden, können Spammer herausfinden, ob eine
E-Mail-Adresse in aktiver Benutzung ist oder nicht -- sprich, ob es sich
lohnt weite Spam-Nachrichten zu senden. Damit diese Maßnahme nicht alle
E-Mails betrifft, verfügen viele E-Mail-Clients über eine
Favoriten-/VIP-Kontakte-Liste, welche die Darstellung von Bildern bei
ausgezeichneten Absendern erlaubt. Alternativ bieten einige Clients
einen Vorschau-Modus an, der E-Mails ohne Bilder oder andere Anhänge
darstellt.

\paragraph{E-Mail-Adresse privat halten}
\label{par:e-mail-adresse-privat-halten}

Spam-Bots können keinen Spam an Adressen senden, die sie nicht kennen.
Für ein öffentliches Auftreten im Internet, insbesondere im Web,
empfiehlt es sich daher eine zweite, öffentliche E-Mail-Adresse zu
nutzen. Diese sollte immer dann verwendet werden, wenn die Adresse
öffentlich auf einer Webseite oder in einer App aufscheint. Da
E-Mail-Adressen ein einfaches Muster aufweisen, können sie gut von
Software-Agenten aufgestöbert werden. Nach diesem System besitzt man
eine private E-Mail-Adresse für Familie, Freunde und Arbeit und eine
öffentliche Adresse für den Rest der Welt. Da wichtige Nachrichten mit
höherer Wahrscheinlichkeit bei der privaten Adresse eingehen, kann der
Spam-Filter für die öffentliche Adresse aggressiver eingestellt werden.
In Kapitel {[}TK 4.1.3{]} wird außerdem beschrieben, wie öffentliche
E-Mail-Adressen besser vor Bots beschützt werden können.

\paragraph{Keine Interaktionen}
\label{par:keine-interaktionen}

E-Mails, die als Spam markiert wurden oder von fremden Absendern
versendet wurden, sollten mit Obacht behandelt werden. Antwortet man auf
eine Spam-Nachricht, erfährt der Spammer, dass die E-Mail-Adresse aktiv
ist. Öffnet man einen Anhang, könnte sich darin Schadsoftware befinden.
Das Klicken auf einen Link kann zu einem der beiden vorherigen Szenarien
führen. Nachrichten, die auf eine dieser Aktionen drängen, sollten
verworfen werden.

\subparagraph{Beispiele}
\label{spar:beispiele}

\begin{itemize}
\tightlist
\item
  \emph{Ihr Account steht kurz vor der Löschung, klicken Sie hier um
  dies zu verhindern.}
\item
  \emph{Sie haben neulich eine Überweisung von 300 Euro vorgenommen,
  klicken Sie hier um dies rückgängig zu machen.}
\item
  \emph{Wie gewünscht finden sie anbei das Dokument über unser
  Gespräch.}
\end{itemize}

\subsubsection{E-Mail-Adressen Verschleierung}
\label{ssub:e-mail-adressen-verschleierung}

Manchmal müssen E-Mail-Adressen auf Webseiten aufscheinen. Sei es auf
der Kontakt-Seite, in den Lizenzvereinbarungen oder im Impressum. Über
die Jahre haben sich verschiedenste Techniken entwickelt,
E-Mail-Adressen auf Webseiten zu verschleiern. Im Folgenden werden drei
Methoden beschrieben, die E-Mail-Adressen für Menschen lesbar und für
Software-Agenten unlesbar machen.

\paragraph{Interpunktion verschleiern}
\label{par:interpunktion-verschleiern}

Das Umschreiben einer Adresse vom Format ›name@example.org‹ in das
Format ›name {[}at{]} example {[}dot{]} org‹ verändert das Muster, nach
dem Software-Agenten suchen. Zusätzlich lassen sich viele weitere
Variationen desselben Tricks erdenken: ›name-AT-example-DOT-org‹. Obwohl
es für Bots nicht unmöglich ist diese Verschleierung zu durchschauen,
ist sie immer noch weitaus besser, als die E-Mail-Adresse ohne
Verschleierung zu veröffentlichen.{[}\^{}TK
\url{http://techblog.tilllate.com/2008/07/20/ten-methods-to-obfuscate-e-mail-addresses-compared/}
2016-05-20{]}

\paragraph{CSS/JavaScipt}
\label{par:cssjavascipt}

Beinahe alle Bots lesen lediglich HTML-Dokumente und ignorieren dabei
externe Ressourcen wie CSS und JavaScript.{[}\^{}TK
\url{http://www.labnol.org/internet/hide-email-address-web-pages/28364/}
2016-05-26{]} Der zusätzliche Aufwand, diese externen Ressourcen
anzufragen, das HTML-Dokument in ein logisches Modell (DOM) umzuwandeln
und folglich den CSS- und JavaScript-Code auf dieses Modell anzuwenden,
ist verhältnismäßig groß und raubt Zeit, die für die Suche anderer
Adressen genutzt werden könnte. Ein mit CSS verstecktes HTML Element,
das inmitten einer E-Mail-Adresse platziert wird, ist eine nahezu
perfekte Verschleierungs-Technik.{[}\^{}TK
\url{http://techblog.tilllate.com/2008/07/20/ten-methods-to-obfuscate-e-mail-addresses-compared/}
2016-05-20{]}

\texttt{html\ \textless{}style\textgreater{}\ {[}data-block\textasciitilde{}=bots{]}\ \{\ display:\ none\ \}\ \textless{}/style\textgreater{}\ \textless{}p\textgreater{}name@\textless{}span\ data-block="bots"\textgreater{}nil\textless{}/span\textgreater{}example.org\textless{}/p\textgreater{}}

Mit JavaScript lassen sich E-Mail-Adressen durch eine Codierung oder
Verschlüsslung vor Bots verbergen. Diese Methoden bieten einen für Bots
praktisch unknackbaren Schutz. Während CSS bei so gut wie 100\% aller
menschlichen Benutzer aktiviert ist,{[}\^{}CSS ist für die Gestaltung
von Webseiten zuständig und bietet daher nicht eine Vielzahl an
Sicherheitslücken, für welche einige Netzer JavaScript deaktivieren. Die
Anzahl an Webseiten-Benutzern, welche CSS deaktiviert haben, ist derart
gering, dass sich zu diesem Thema keine fundierte Statistiken finden
lasse.{]} ist JavaScript nur bei 98\% aktiviert.{[}\^{}TK
\url{https://developer.yahoo.com/blogs/ydn/many-users-javascript-disabled-14121.html}
2016-05-21{]} JavaScript basierte Verschleierungen grenzen damit 2\%
aller Benutzer aus, erlauben jedoch eine beliebige Verschlüsslung der
Adresse, darunter zum Beispiel ROT13.{[}\^{}TK Christopher Swenson,
Modern Cryptanalysis: Techniques for Advanced Code Breaking. John Wiley
\& Sons. S. 5, ISBN 9780470135938{]}

\paragraph{HTML-Codierung}
\label{par:html-codierung}

HTML unterstützt selbst eine Reihe von Codierungen,{[}\^{}Ein Generator
für codierte E-Mail-Adressen findet sich unter
\url{http://rumkin.com/tools/mailto_encoder/custom.php}{]} die auf
E-Mail-Adressen angewandt werden kann.

\subparagraph{URL-Codierung}
\label{spar:url-codierung}

Bei einer URL Codierung werden einzelne Zeichen durch ein Prozentzeichen
gefolgt von zwei alphanumerischen Zeichen ersetzt. Die Kombination der
beiden alphanumerischen Zeichen lässt sich in standardisierten Tabellen
nachschlagen.{[}\^{}TK \url{https://tools.ietf.org/html/rfc3986}
2016-06-05{]}

\texttt{html\ \textless{}a\ href="mailto:\%6e\%61\%6d\%65\%40\%65\%78\%61\%6d\%70\%6c\%65\%2e\%6f\%72\%67"\textgreater{}name@example.org\textless{}/a\textgreater{}}

\subparagraph{HTML-Entitäten-Referenzen}
\label{spar:html-entitaeten-referenzen}

Eine Zeichen-Entität-Referenz beschreibt ein Zeichen, indem der
Code-Point link von \texttt{\&\#} oder \texttt{\&\#x} und rechts von
\texttt{;} umgeben sein muss. Der Code-Point beschreibt die numerische
Position, an welcher das Zeichen im Unicode-Standard geführt wird.
Beginnt die Entitäten-Referenzen mit \texttt{\&\#x}, muss der Code-Point
in Hexadezimalzahlen, anstelle von Dezimalzahlen, angegeben werden.

\texttt{html\ \textless{}a\ href="mailto:\&\#110;\&\#97;\&\#109;\&\#101;\&\#64;\&\#101;\&\#120;\&\#97;\&\#109;\&\#112;\&\#108;\&\#101;\&\#46;\&\#111;\&\#114;\&\#103;"\textgreater{}name@example.org\textless{}/a\textgreater{}}

\subparagraph{URL-Codierung + HTML-Entitäten-Referenzen}
\label{spar:url-codierung-html-entitaeten-referenzen}

\texttt{html\ \textless{}a\ href="mailto:\%6e\%61\&\#109;\&\#101;\&\#64;\&\#101;\&\#120;\&\#97;\&\#109;\%70\%6c\&\#101;\%2e\&\#111;\%72\&\#103;"\textgreater{}name@example.org\textless{}/a\textgreater{}}

\subsubsection{Kommentare}
\label{ssub:kommentare}

Neben E-Mails stellen online-Kommentare ein weiteres Medium für
Spam-Nachrichten dar. Um Kommentare vor Bots zu schützen, lassen sich
verschiedene Techniken zum Einsatz bringen. Dazu zählen CAPTCHAs,{[}TK
REF 4.3.2{]} Honeypots,{[}TK REF 4.3.3{]} die Pflicht, sich beim
Kommentarsystem anzumelden und Zwei-Schritt-Absenden-Verfahren.

Im Folgenden wird eine Spam-Attacke beschrieben, die auf ein
Kommentarsystem vom 6. Mai 2015 (20:57 Uhr) bis zum 7. Mai 2015 (06:21
Uhr) durchgeführt wurde. Das Kommentarsystem befand sich unter der
Adresse
\texttt{*.blogfill.de/weblog/funktionsweise-von-easing-funktionen} und
setzte als einzige Anti-Spam-Technik eine Kommentarvorschau ein, die vor
dem Absenden eines Kommentars durchgeführt werden musste. Das alleine
bot bereits einen guten Schutz gegen Spam,{[}\^{}TK
\url{http://textpattern.com/faq/82/how-do-i-skip-the-comment-preview}
2016-05-28{]}{[}\^{}TK \url{https://vimeo.com/5397743} 2016-05-28{]}
verhinderte jedoch nicht, dass im genannten Zeitraum 719 Spam-Kommentare
eingereicht wurden.

Bei den 719 Kommentaren wurde der Name „Walter“ siebenmal und damit am
häufigsten gewählt. Es wurden insgesamt 71 Links publiziert, dabei
handelt es sich um 35 voneinander unterschiedliche Exemplare. Die drei
häufigsten Links wurden jeweils viermal publiziert, einer davon
verwendete HTTPS im Gegensatz zu HTTP. Aus den 35 unterschiedlichen
Links verwendeten nur zwei HTTPS, beide mit demselben Hostname:
\texttt{ummgc.org}. Die Verwendung von HTTPS ist insofern interessant,
als dass HTTPS ein Zertifikat verlangt, welches von einer zumeist
externen Zertifizierungsstelle{[}\^{}Sind Zertifikate extern?{]}
ausgestellt werden muss. Spammer können so leicht rückverfolgt werden.

Auch interessant ist es, dass von insgesamt 376.165 Zeichen 0,053~\%
Steuerzeichen sind, also nicht normale Zeichen, die am Bildschirm
dargestellt werden, sondern die Darstellung anderer Zeichen
beeinflussen. Solche seltene Zeichen könnten eingesetzt werden um in
primitiven Systeme, die nicht mit solcherlei Zeichen rechnen, einen
Fehler zu verursachen. 0,27~\% aller Zeichen machen Währungszeichen aus,
davon 914 Euro-, 91 Dollar- und 22 Pfund-Zeichen.

Ein vollständiger Report aller Daten und eine Beschreibung, wie sie
gewonnen wurden, findet sich online unter
\url{https://github.com/SoftwareAgenten/GA-Archive/tree/master/Comments}.
  \subsection{Phishing} % (fold)
\label{sub:phishing}

% subsection phishing (end)
  \subsection{Künstliche Intelligenz} % (fold)
\label{sub:kunstliche_intelligenz}
Die ersten Software-Bots waren noch weit entfernt vom selbstständigen Verhalten
der heutigen Software-Agenten. Sie wurden programmiert um eine einzige Aufgabe
zu erledigen.\footcite{wa:spamFilter} Spam-Emails an eine
vordefinierte Liste von Empfängern zu senden, zum Beispiel. Diese
Verfahrensweise skaliert nicht in die Dimensionen, die das Internet inzwischen
angenommen hat. Zusammen mit dem Internet sind Computer immer schneller,
komplexer und leistungsfähiger geworden. So lassen sich heute autonome
Anwendungen realisieren, die selbst auf der Suche nach E-Mail-Adressen das WWW
durchstöbern, zur Situation passende Texte verfassen, diese versenden und auf
Antworten warten.

Das Motto eines voll-autonomen Software-Agenten ist es, den Menschlichen Faktor
bei der Beschaffung von Daten zu eliminieren. Wie oben beschrieben sind Bots
besonders dazu geeignet Listen von Daten anzulegen und diese abzuarbeiten.
Manchmal wird das von Benutzern des Internets geduldet -- wie bei der
Indexierung von öffentlichen Webseiten -- und manchmal nicht -- wie bei der
Aufspürung von privaten Daten.

Um Bots von gewissen Orten im Internet auszusperren wurden verschiedene
Techniken entwickelt, darunter CAPTCHAs, Honeypots, und der Zwang zur
Beglaubigung als Mensch. Die ersten beiden Techniken werden in diesem Kapitel
noch genauer behandelt. Beim Zwang zur Beglaubigung als Mensch handelt es sich
um das Vorweisen von persönlichen Daten, die den Benutzer als Menschen
authentifizieren. Dazu zählen:

\begin{itemize}
\item
  Ein Nachrichten-Dienst, der seine Nachrichten über das Internet versendet,
  demgegenüber die Handynummer bei der Registrierung eines neuen Accounts
  verlangt. Die Handynummer dient in diesem Beispiel als rares Gut, das ein Bot
  nicht beliebig beschaffen kann.
\item
  Ein Online-Shop, der bereits vor der ersten Abbuchung nach
  Kreditkarteninformationen fragt.
\item
  Eine Online-Gemeinschaft, die beim Beitritt eines neuen Mitglieds eine
  digitale Kopie des Personalausweises verlangt.
\end{itemize}

\subsubsection{CAPTCHA} % (fold)
\label{ssub:captcha}
Eine aktive Software-Bot Blockade wird als CAPTCHA (für \emph{Completely
Automated Public Turing Test To Tell Computers and Humans Apart},
\footcite{captchaNet} zu Deutsch: \emph{Vollständig automatisierter,
öffentlicher Turing-Test zur Unterscheidung von Computern und Menschen})
bezeichnet. Exemplarische Einsatzgebiete sind die Reduzierung von
Spam-Nachrichten durch ein online Kontaktformular und der Ausschluss von Bots
bezüglich der Registrierung bei einem sozialen Netzwerk.

Verbreitete Ausführungen der CAPTCHA-Methode, welche für diese Arbeit von
Wichtigkeit sind, umfassen:

\begin{description}
  \item[Verzerrter Text]
  Der Benutzer muss eine verzerrte Abbildung einer Zeichenfolge in ein Textfeld
  eingeben. Die Verzerrung ist stark genug, als dass sie im Idealfall nicht von
  Software ausgemacht werden kann und dabei dennoch für Menschen lesbar bleibt.
  
  \item[Objekt Klassifikation]
  Dem Benutzer wird eine Gruppe von Bildern präsentiert, von welchen jene
  markiert werden müssen, welche zu einer bereitgestellten Beschreibung passen.
  
  \item[Akustisches Diktat]
  Verschiedene Stimmen diktieren dem Benutzer einen alphanumerischen Code der
  in ein Textfeld eingegeben werden muss. Diese Methode ergänzt zumeist andere
  Ausführungen der CAPTCHA-Methode um Rücksicht auf visuell eingeschränkte
  Benutzer zu nehmen.
\end{description}
% subsubsection captcha (end)

\subsubsection{Honeypot} % (fold)
\label{ssub:honeypot}
Als Honeypot (zu Deutsch \emph{Honigtopf}, sinngemäß \emph{Fettnäpfchen}) wird
in der Informatik eine Falle bezeichnet, die sich als begehrenswertes Objekt
tarnt.

Ein exemplarischer Einsatz ist das von Ordnungshütern durchgeführte
Veröffentlichen von scheinbar illegalen Dateien im Internet. Nutzer, welche auf
diese Dateien zugreifen, werden vom Honeypot registriert und können daraufhin
strafrechtlich verfolgt werden.

Dieselbe Methodik lässt sich bei der Unterscheidung zwischen Bots und Menschen
zum Einsatz bringen. Eine geläufige Praxis ist es, online Formulare (zum
Beispiel ein Kommentar-Formular) mit einem zusätzlichen Textfeld zu bestücken
und dieses visuell zu verbergen. Ein menschlicher Benutzer kann dieses Feld
weder erreichen noch wahrnehmen; konträr stoßen Software-Agenten auf das
Textfeld – sie nehmen schließlich nur den Programmcode des Formulars und nicht
die visuelle Aufmachung wahr – und füllen es aus. Beim Absenden des Formulars
kann die Formular-Software überprüfen, ob das Honeypot-Feld ausgefüllt wurde
oder nicht; im ersten Falle würde diese Software das Absenden unterbinden.
% subsubsection honeypot (end)
% subsection kunstliche_intelligenz (end)
  % section beobachten_und_verstehen (end)
  
  \section{Eigener Software-Agent} % (fold)
  \label{sec:eigener_software_agent}
  \subsection{Maschinelles Lernen: CAPTCHA} % (fold)
\label{sub:maschinelles_lernen_captcha}

% subsection maschinelles_lernen_captcha (end)
  % section eigener_software_agent (end)
  
  \onecolumn \printbibliography
\end{document}