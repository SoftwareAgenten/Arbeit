\subsection{Methoden}
\label{sub:forschung_methoden}

Die Informationen über Bots im Internet stützen sich zu großen Teilen auf
eigene Forschung, die im Rahmen dieser Arbeit betrieben wurde. Dafür wurden
zwei unabhängige, wenn auch zueinander kompatible, Softwares entwickelt. Die
erste, \emph{Great-Attractor} (Kapitel \ref{topic:great-attractor}), ist ein
Bot-Beobachtungs-System, das in eine Webseite eingebunden werden kann. Teil der
\emph{Great-Attractor}-Software ist der \emph{GA-Analyzer}, welcher das
Auswerten der von \emph{Great-Attractor} gesammelten Daten in einer
interaktiven Shell erlaubt. Die zweite Software ist WWWEB (Kapitel
\ref{sub:wwweb}), ein Software-Agent, der auf den Beobachtungen der Kapitel
\ref{sec:beobachten_und_verstehen} und
\ref{sub:auswertung-der-great-attractor-daten} basiert.

\emph{Great-Attractor}, \emph{GA-Analyzer}, WWWEB und andere Programme, die im
Rahmen dieser Arbeit entstanden sind, finden sich online unter
\url{https://github.com/SoftwareAgenten/}. Sie wurden unter der MIT
Lizenz\footcite{license:MIT} veröffentlicht und können daher für andere,
persönliche und kommerzielle Arbeiten eingesetzt werden. Entsprechende
Dokumentation wurde in Form von README.md Dateien bereitgestellt, welche die
Funktionsweise der einzelnen Programme erläutern.
