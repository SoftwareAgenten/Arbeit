\subsection{WWWEB}
\label{sub:wwweb}

Dieses Kapitel beschreibt anhand eines konkreten Beispiels, wie ein
Software-Agent aufgebaut ist und wie er funktioniert. Eigens für diese Arbeit
wurde ein Webcrawler namens WWWEB\footnote{Der Quellcode für WWWEB findet
sich unter \url{https://github.com/SoftwareAgenten/WWWEB}} (kurz für
\emph{World Wide Web Exploring Bot}, englisch für \emph{Weltweites Netz
erkundender Roboter}) programmiert. WWWEB startet an einer vordefinierten
Domäne, zum Beispiel bei \texttt{example.org} und durchsucht die dazugehörige
Webseite auf Port 80 nach allen Links, die auf andere Domänen zeigen,
beispielsweise \texttt{other.org} und \texttt{another-one.org}. Wurden diese
Domänen gefunden, macht WWWEB eine Anfrage an
\texttt{http://example.org/robots.txt}. (Siehe \ref{robotstxt}.) Diese Datei wird
folglich analysiert und in einem abstrakten Datenmodell festgehalten.

Damit hat WWWEB \texttt{example.org} abgearbeitet. Die Domäne wird in eine
Liste eingetragen, um nicht erneut analysiert zu werden. Im Folgenden
durchläuft WWEB dieselbe Prozedur auf einer der zuvor gefundenen Domänen. Da
die meisten Startseiten zu mehr als einer anderen Domäne verlinken, füllt sich
die Liste schneller an, als sie entleert wird.

Periodisch werden Berichte im JSON-Dateiformat angelegt, um beim Beenden des
WWWEB Programmes möglichst wenige Daten zu verlieren. Ursachen für ein solches
Beenden können sein:

\begin{itemize}
\item
  WWWEB wurde vom Betriebssystem beendet, um Ressourcen frei zu machen.
\item
  Der Host, auf welchem WWWEB läuft, hat eine begrenzte Laufzeit für
  \emph{Node.js}-Prozesse\footnote{WWWEB wurde für die Software-Plattform
  \emph{Node.js} programmiert.} festgelegt.
\item
  Der Computer wurde von der Stromversorgung getrennt.
\end{itemize}

\subsubsection{Architektur}
\label{ssub:wwweb_architektur}

Das Programm läuft großteils in einer Haupt-Routine, die kontinuierlich von der
\texttt{crawl}-Funktion aufgerufen wird. \texttt{crawl} verwaltet einen
Stapelspeicher (\emph{Stack}), auf welchen gefundene Domänen gelegt und auch
wieder heruntergenommen werden können. Die Domänen werden deshalb nach dem
\emph{Last Come, First Served} (\emph{zuletzt herein -- zuerst hinaus}) Prinzip
behandelt. Die Summe aller untersuchen Domänen wird als Bericht (\emph{Report})
bezeichnet.

Ein Bericht besteht aus einer Startzeit, der Startdomäne und einem Protokoll.
Ein Protokoll wiederum besteht aus einer Reihe von Aufzeichnungen
(\emph{Records}). Diese Aufzeichnungen setzen sich aus einem Zeitpunkt und
einer Domäne-Konfiguration (\emph{DomainSetting}) zusammen, welche eine Domäne
mit einem Satz an Regeln (\emph{RuleSet}) verknüpfen. Ein \emph{RuleSet}
beschreibt den Zusammenhang eines User-Agents und den dafür definierten Regeln
(\emph{Rules}).

WWWEBs Netzwerkanfragen setzen auf \emph{Promises} auf, eine Technik, bei
welcher ein Teil des Programms ein Versprechen äußert, gewisse Daten
bereitzustellen. So lässt sich ein asynchroner Programmierstil mit iterativen
Charakteristiken umsetzen.

Der Speichervorgang, welcher den aktuellen Bericht in einen JSON-String
umwandelt und in das Dateisystem schreibt, wird periodisch von der
Event-Schleife der \emph{Node.js}-Umgebung aufgerufen und ist damit unabhängig
vom Kontrollfluss des restlichen Programms.

\subsubsection{Installation}
\label{ssub:wwweb_installation}

WWWEB setzt die \emph{Node.js}-Umgebung in der Version 6.0.0 oder höher voraus
und lässt sich über den Package-Manager \emph{npm} installieren.\footnote{Die
Projektseite für das npm-Packet findet sich unter
\url{https://www.npmjs.com/package/wwweb}} Sowohl \emph{Node.js} als auch
\emph{npm} sind für Windows, OS~X und Linux verfügbar.

Sind \emph{Node.js} und \emph{npm}, jeweils in der erforderten Version,
vorhanden, so genügt der folgende Befehl, um WWWEB auf einem Computer zu
installieren.

\begin{lstlisting}[language=shell]
$ npm install -g wwweb
\end{lstlisting}

\subsubsection{Einsatz}
\label{ssub:wwweb_einsatz}

Bei WWWEB handelt es sich um ein Kommandozeilen-Tool. Es verlangt zwei
Parameter: die Startdomäne und das Ausgabeverzeichnis, in welches die Berichte
geschrieben werden. Weitere Parameter können mit \texttt{wwweb\ -\/-help}
eingesehen werden.

Der folgende Befehl startet WWWEB bei \texttt{example.org}, schreibt alle zwei
Minuten einen Bericht in das Verzeichnis \texttt{\textasciitilde{}/reports} und
wartet höchstens neun Sekunden auf das Laden von Dateien.

\begin{lstlisting}[language=shell]
$ wwweb -d example.org -s 120 -o ~/reports --timeout 9000
\end{lstlisting}

Für einen besseren Einblick in die Funktionsweise des Programms bietet WWWEB
einen zweistufigen Verbose-Modus\footcite{wp:verbose} an. Mit \texttt{-v}
werden Warnmitteilungen und unwichtige Informationen ausgegeben. Ist der
Parameter doppelt vorhanden (\texttt{-vv}) werden ebenfalls
Debugging-Informationen ausgegeben.
