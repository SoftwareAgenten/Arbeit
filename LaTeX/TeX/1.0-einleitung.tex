\section{Einleitung}
\label{sec:einleitung}

\subsection{Motivation}
\label{sub:motivation}

Das Internet wird ein immer existenziellerer Bestandteil des modernen Lebens.
Waren es anfangs nur wenige Standcomputer, die an diesem Netzwerk teilhaben
durften, so wird heute die Zahl von Internet-vernetzten Geräte auf 13,4
Milliarden geschätzt;\footcite{iotTripleBy2020} für das Jahr 2020 wird bereits
eine Summe von 38,5 Geräten prognostiziert.\footcite{iotTripleBy2020} Um eine
Welt zu verstehen, in der es doppelt so viele miteinander vernetzte Geräte als
Menschen gibt, muss man Software-Agenten verstehen. Sie sind die unsichtbaren
Arbeiter des Internets. Software-Agenten verknüpfen Webseiten, versenden
Spam-Nachrichten, schreiben Wikipedia-Artikel\footcite{botWritingForWikipedia}
und vieles mehr. Ohne Sie hätte das Internet nie die Größe erreicht, die es
heute hat. Längst reicht es über Standcomputer hinaus: Smartphones, Konsolen,
Armbanduhren, Kühlschränke und Autos, alles Produkte die das \emph{Internet of
Things} miteinander vernetzt. So spielt das Internet eine essenzielle Rolle in
der Wirtschaft, der Politik, der Arbeit und der Freizeit. Wer Software-Agenten
versteht ist besser über genannte Bereiche informiert und kann leichter in
deren Zukunft blicken.

\subsection{Aufbau}
\label{sub:aufbau}

Diese Arbeit beginnt mit einem Überblick über den aktuellen Stand zum Thema
Software-Agenten im Internet, bei dem die Einsatzgebiete von
Bots\footref{sub:einsatz-von-software-agenten} beschrieben werden. Dabei wird
das Thema Spam im Detail erarbeitet,\footref{sub:spam} bei welchem Techniken
zur E-Mail-Spam-Vermeidung\footref{ssub:e-mail-spam-vermeiden} und die
Spam-Kommentar Problematik\footref{ssub:kommentare} besprochen werden.
Anschließend wird der Aufschwung der Künstlichen Intelligenz
besprochen\footref{sub:kunstliche_intelligenz} und wie dieser das
Machverhältnis zwischen Software-Agenten und Menschen verändert hat und
verändern wird. Der zweite Teil der Arbeit wendet sich der Praxis zu und
beschreibt das Beobachten fremder
Software-Agenten\footref{sec:die_software_agenten_suite} sowie das Erstellen
eines eigenen Bots.\footref{sub:wwweb}
