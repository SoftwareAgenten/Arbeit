Die Daten über das Browse-Verhalten von Software-Agenten wurden von einer
eigens für diese Arbeit erstellten Software-Suite gesammelt. Die Suite wurde
unter dem Namen \emph{Great Attractor}\footnote{Der Name wurde in Anlehnung an
das kosmische Filament, im deutschsprachigen Raum auch als \emph{Großer
Attraktor} bekannt, gewählt.} im Internet unter der MIT-Lizenz\footnote{Eine
Kopie der Lizenz ist unter \url{https://legal.addpixel.net/licenses/MIT/} zu
finden.} auf der Open-Source Code Plattform GitHub.com
veröffentlicht.\footnote{GitHub.com Projektseite:
\url{https://github.com/SoftwareAgenten/Great-Attractor}}

Die MIT-Lizenz erlaubt die Vervielfältigung, Veränderung und Unterlizenzierung
der \emph{Great Attractor} Softwares sowohl für den privaten als auch den
kommerziellen Gebrauch. Dabei muss die Lizenzangabe samt Copyright Anmerkung
unverändert beibehalten werden.\footnote{Hierbei handelt es sich lediglich um
eine kurze Zusammenfassung die keinerlei Anspruch auf Vollständigkeit oder
Rechtsgültigkeit ausübt. Eine vollständige Kopie der Lizenz findet sich unter
\url{https://legal.addpixel.net/licenses/MIT/}.} Zufolge der Open Source
Initiative ist die MIT-Lizenz der Open Source Definition
konform.\footcite{license:MIT} \emph{Open Source} bedeutet in diesem Kontext:
der Quellcode der \emph{Great Attractor} Softwares kann frei eingesehen werden
und in anderen Software-Projekten zum Einsatz kommen.

GitHub.com wurde zugrunde der folgenden Argumente als Open-Source Code
Plattform gewählt.

\begin{enumerate}
\item
  GitHub.com ist zur Zeit dieses Schreibens die populärste Open-Source Code
  Plattform\footcite{githubPop} und erfreut sich demgemäß einer großen Anzahl an
  aktiven Nutzern.
\item
  GitHub setzt auf der \emph{git} Versionsverwaltungssoftware auf, welche sowohl
  bei den \emph{Great Attractor} Softwares zum Einsatz kommt als auch bei der
  Erstellung der \emph{Software-Agenten im Internet} Arbeit verwendet wird.
\end{enumerate}

\subsection{Architektur und Philosophie}
\label{sub:architektur_und_philosophie}
\emph{Great Attractor} (GA) besteht aus zwei Kernkomponenten: dem gleichnamigen
Archiv an von Bots produzierten Daten\footnote{vollständig einsichtbar unter
\url{https://github.com/SoftwareAgenten/Great-Attractor}} und eine Reihe an
Applikationen, die Datensätze für das Archiv sammeln.\footnote{Eine Liste alle
Applikation kann unter
\url{https://github.com/SoftwareAgenten/Great-Attractor\#data-collectors}
eingesehen werden.}

Das GA-System erfüllt folgende Anforderungen.

\begin{itemize}
\item
  \textbf{Portabilität der Daten}. Sämtliche durch GA gesammelte Daten werden im
  JSON-Format auf dem Dateisystem gesichert. Dies erleichtert es
  Sicherungskopien der Datensätze anzufertigen und diese folglich zu
  veröffentlichen bzw. auf anderen Wegen zu vervielfältigen.
\item
  \textbf{Portabilität des Systems}. GA wurde in der Programmiersprache PHP
  verfasst, welche auf allen gängigen Computer-Betriebsystemen\footnote{
  Darunter: \emph{Linux}, \emph{FreeBSD}, \emph{Unix}, \emph{Windows}
  und \emph{Darwin}} genutzt werden kann. Außerdem ist das JSON
  Datenaustauschformat, in welchem Daten gesichert werden, einfach für Menschen
  zu lesen\footcite{jsonDe} und für Maschinen einfach zu
  parsen.\footcite{jsonDe} Dies erlaubt es nicht-GA-Systemen auf die gewonnenen
  Daten zurückzugreifen und ergänzende Operationen auszuführen.
\end{itemize}

\subsection{Funktionsweise der GA-Softwares}
\label{sub:funktionsweise_der_ga_softwares}
Das GA-System sammelt Daten über das Verhalten von Webseiten-Besuchern. Um
diese Daten zu erheben, muss es in eine bestehende Webseite eingebunden werden
die folglich als Wirt dient. Sämtliche Logik und Konfiguration einer
GA-Software findet sich in einer einzigen PHP-Datei: \texttt{system.php}. Dies
erleichtert es, die GA-Software in eine Webseite einzubinden. Die Datei
befindet sich in einem Verzeichnis namens \texttt{great-attractor}, welches als
Namensraum\footcite{wp:namensraum} dient. Aufgezeichnete Daten werden in ein
Verzeichnis namens \texttt{data} als JSON-Dateien geschrieben. Sollte dieser
Name mit dem bestehenden Webseite-System in Konflikt

\subsubsection{Installation}
\label{ssub:ga_installation}
Es empfiehlt sich, das \texttt{great-attractor} Verzeichnis im Root-Verzeichnis
der Wirt-Webseite zu platzieren. Dies erleichtert die Installation der
Software. Sollte die Wirt-Webseite ebenso PHP als Skript-Sprache nutzen, so
lässt sich das GA-System wie folgt einbinden.

\begin{lstlisting}[language=PHP]
<?php

# GA

include_once('great-attractor/system.php');

ga_init('tracking-name');
ga_register_visit();
ga_register_request();

if (!empty($_POST)) {
  ga_register_form_data($_POST);
}

# END GA
\end{lstlisting}

Obiger PHP-Code bindet das System ein (\texttt{include\_once}), legt den Namen
der Webseite fest (\texttt{ga\_init}), registriert den Besuch
(\texttt{ga\_register\_visit}), registriert die HTTP-Anfrage
(\texttt{ga\_register\_request}) und, sollte ein Formular mit HTTP-POST-Daten
abgesendet worden sein, registriert das POST Objekt
(\texttt{ga\_register\_form\_data}).

Falls das Webseite-System nicht PHP nutzt, muss manuell eine Software-Brücke
zum GA-System gebaut werden, worüber die oben genannten Funktionen aufgerufen
werden.

\subsubsection{Konfiguration}
\label{ssub:ga_konfiguration}
Innerhalb der Datei \texttt{system.php} findet sich die \texttt{\$paths}
Variable. Sie listet sämtliche Verzeichnisse, die für das GA-System notwendig
sind. Verzeichnisse, die in \texttt{\$paths} aufgeführt werden, nicht aber
Bestandteil des Dateisystems sind, werden bei der ersten Ausführung der
GA-Software automatisch angelegt.

\subsubsection{Verhalten}
\label{ssub:verhalten}
Das GA-System kann drei verschiedene Arten von Daten erheben: Die Anzahl der
Besuche (wobei eine Sitzung als ein Besuch gezählt wird), der HTTP-Header einer
Anfrage und die Einzelheiten einer HTTP-POST-Anfrage (die Daten eines
abgesendeten Web-Formulars).

\paragraph{Besuche-Zähler}
\label{par:ga_besuche_zahler}
Die Funktion \texttt{void\ ga\_register\_visit()} erhöht den Besuche-Zähler für
eine gegebene Webseite. Der Name der Webseite wird durch \texttt{void\
ga\_init(\$page\_name)} festgelegt. Wiederholte Aufrufe derselben Sitzung
werden nicht als weitere Besuche gezählt.

Ist der HTTP-GET-Parameter \texttt{thisisanadmin} auf 1 gesetzt wird der
Besuch, sowie Folgebesuche derselben Sitzung, nicht von der Software erfasst.

\paragraph{HTTP-Header}
\label{par:ga_http_header}
Mit der Funktion \texttt{void\ ga\_register\_request()} sichert GA den
HTTP-Request-Header und weitere Kenndaten der Anfrage. Darunter finden sich:

\begin{itemize}
\item
  Name der Webseite (\texttt{pageName})
\item
  aktuelle Uhrzeit samt Zeitzone (\texttt{time}, \texttt{timezone})
\item
  vollständige URL (\texttt{url})
\item
  Referenz auf die Quelle der Anfrage (\texttt{reference})
\item
  IP-Adresse des Besuchers (\texttt{ipAddress})
\item
  eindeutige ID, die der Sitzung zugewiesen wurde (\texttt{userId})
\item
  eindeutige ID, die die Anfrage beschreibt (\texttt{id})
\item
  eine Liste aller vorhergehenden Anfrage-IDs der Sitzung
  (\texttt{previousRequestIds})
\item
  diverse Informationen über die Geoposition des Besuchers, darunter: Stadt,
  Region und Ländercode (\texttt{geo})
\end{itemize}

\paragraph{POST-Daten}
\label{par:ga_post_daten}
Die Funktion \texttt{void\ ga\_register\_form\_data(\$post)} nimmt ein PHP-POST
Objekt (für gewöhnlich \texttt{\$\_POST}) entgegen und sichert es als
JSON-Datei. Weiters hinzugefügt werden: die Zeit (\texttt{time}), die Zeitzone
(\texttt{timezone}), der Namen der Webseite (\texttt{pageName}), die ID der
Sitzung (\texttt{userId}), die ID der Anfrage (\texttt{requestId}) und der
Dateiname der Anfrage (\texttt{requestFilename}).

\subsubsection{Einsatz}
\label{ssub:ga_einsatz}
Nachdem eine GA-Software in eine Wirt-Webseite eingebunden wurde, muss diese
online veröffentlicht werden. Die bloße Existenz einer Webseite garantiert
nicht für Besuche von Software-Agenten. Es empfiehlt sich daher Links zur
Webseite auf diversen Online-Plattformen zu publizieren. Damit die einzelnen
Plattformen, also die Quellen der Aufrufe, von der GA-Software zuverlässig
identifiziert werden können, muss der Link auf die Webseite den GET-Parameter
\texttt{r} bei sich führen, der die Quelle beschreibt.

Beispiel: die Wirt-Webseite befindet sich unter www.example.org und ein Link
dorthin soll auf der Plattform www.twitter.com publiziert werden. Demnach
könnte der auf www.twitter.com publizierte Link wie folgt gestaltet werden:
www.example.org/?r=tw.

Außerdem empfiehlt es sich eine Liste zu führen, auf der das Datum und die Zeit
der Publizierung auf den einzelnen Plattformen verzeichnet wird. So lassen sich
bessere Rückschlüsse über das Verhalten der Besucher ziehen.
