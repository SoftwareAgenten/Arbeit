\subsection{Spam} % (fold)
\label{sub:spam}
Unerwünschte Nachrichten im Internet, größtenteils Werbeangebote,
Phishing-Attacken oder Übermittler von Schadsoftware, werden als Spam
bezeichnet. Spam-Nachrichten stellen per definitionem eine leidige
Kommunikationsform dar. Die gewöhnlichste Art, namhaft durch ihre Omnipräsenz
im Leben mit dem Internet, stellt der E-Mail-Spam dar.

Der Begriff fand noch vor dem Erscheinen des \emph{World Wide Web} Anklang,
beschrieb damals jedoch eine Flut an unerwünschten
Nachrichten.\footcite{originTermSpam} Erste Aufkommen des Phänomens wurden im
ARPANET\footcite{rfc:706} und im USENET beobachtet.

Der Name \emph{Spam} entspringt dem gleichnamigen Dosenfleisch, welches im
Zweiten Weltkrieg in großen Mengen an Soldaten verteilt
wurde.\footcite{lifeDuringSecondWW} Der Markenname Spam® (kurz für \emph{Spiced
Ham}) setzte sich auf diese Weise rasch im Vereinigten Königreich als Deonym
für Frühstücksfleisch durch. Die britische Comedy-Gruppe Monty Python griff im
Sketch \emph{Spam}\footcite{spamMontyPython} der BBC Serie \emph{Monty Python’s
Flying Circus} die Allgegenwärtigkeit des Fleischprodukts auf, weshalb dieses
bis heute als Symbol für einen unerwünschten Überschuss fungiert.

Mit dem Begriff \emph{Spam} werden heutzutage diverse Arten von Nachrichten
bezeichnet. Werbenachrichten sind das klassische Beispiel einer Spam-Nachricht.
Sie bieten zumeist dubiose Produkte oder Dienstleistungen an.

\subsubsection{E-Mail-Spam vermeiden}
\label{ssub:e-mail-spam-vermeiden}

Das Problem, Werbung und andere unerwünschte Nachrichten zu erhalten, ist nicht
neu; Papier-Post belästigt Hausbesitzer seit Generationen. Gegenmaßnahmen wie
\enquote{Bitte keine Werbung einwerfen}-Kleber für das Postfach sind etablierte
Präventivmaßnahmen, die in einigen Staaten gesetzlich verankert
sind.\footcite{someReklameIllegal} Digitale Spam-Nachrichten dagegen sind ein
relativ neues Phänomen und viele E-Mail-Nutzer fühlen sich durch eine Flut von
Spam-Nachrichten überlastet. Die folgenden Ratschläge empfehlen sich, um
E-Mail-Postfächer vor Spam zu schützen.

\paragraph{Spam-Filter}
\label{par:spam-filter}

Spam-Filter identifizieren und markieren Spam-Mails. Sie können direkt auf
einem Mail-Server oder in einem E-Mail-Client installiert werden. Als Spam
markierte Mails werden dabei in ein eigenes Postfach bewegt; sollten wichtige
E-Mails nicht im Posteingang erscheinen, kann man das Spam-Postfach auf die
erwarteten Nachrichten untersuchen. Viele E-Mail-Clients erlauben das manuelle
Markieren von Spam-Mails, sollte eine E-Mail den Spam-Filter überlisten. Diese
Aktion ist dem einfachen Löschen der Nachricht insofern überlegen, als dass der
Client die markierte E-Mail auf Spam-Indikatoren untersucht um diese in
seinen Spam-Filter-Algorithmus zu integrieren.

\paragraph{Bilder unterbinden}
\label{par:bilder-unterbinden}

Eine drastische, wenn auch effiziente Maßnahme, ist das vollständige Blockieren
aller Bilder. Dadurch, dass Bilder erst beim Darstellen einer E-Mail geladen
werden, können Spammer herausfinden, ob eine E-Mail-Adresse in aktiver
Benutzung ist oder nicht -- sprich, ob es sich lohnt weite Spam-Nachrichten zu
senden. Damit diese Maßnahme nicht alle E-Mails betrifft, verfügen viele
E-Mail-Clients über eine Favoriten-/VIP-Kontakte-Liste, welche die Darstellung
von Bildern bei ausgezeichneten Absendern erlaubt. Alternativ bieten einige
Clients einen Vorschau-Modus an, der E-Mails ohne Bilder oder andere Anhänge
darstellt.

\paragraph{E-Mail-Adresse privat halten}
\label{par:e-mail-adresse-privat-halten}

Spam-Bots können keinen Spam an Adressen senden, die sie nicht kennen. Für ein
öffentliches Auftreten im Internet, insbesondere im Web, empfiehlt es sich
daher eine zweite, öffentliche E-Mail-Adresse zu nutzen. Diese sollte immer
dann verwendet werden, wenn die Adresse öffentlich auf einer Webseite oder in
einer App aufscheint. Da E-Mail-Adressen ein einfaches Muster aufweisen, können
sie gut von Software-Agenten aufgestöbert werden. Nach diesem System besitzt
man eine private E-Mail-Adresse für Familie, Freunde und Arbeit und eine
öffentliche Adresse für den Rest der Welt. Da wichtige Nachrichten mit höherer
Wahrscheinlichkeit bei der privaten Adresse eingehen, kann der Spam-Filter für
die öffentliche Adresse aggressiver eingestellt werden. In Kapitel
\ref{ssub:e-mail-adressen-verschleierung} wird außerdem beschrieben, wie
öffentliche E-Mail-Adressen besser vor Bots beschützt werden können.

\paragraph{Keine Interaktionen}
\label{par:keine-interaktionen}

E-Mails, die als Spam markiert oder von fremden Absendern versendet wurden,
sollten mit Vorsicht behandelt werden. Antwortet man auf eine Spam-Nachricht,
erfährt der Spammer, dass die E-Mail-Adresse aktiv ist. Öffnet man einen
Anhang, könnte sich darin Schadsoftware befinden. Das Klicken auf einen Link
kann zu einem der beiden vorherigen Szenarien führen. Nachrichten, die auf eine
dieser Aktionen drängen, sollten gelöscht werden.

\subparagraph{Beispiele}
\label{spar:beispiele}

\begin{itemize}
\item
  \emph{Ihr Account steht kurz vor der Löschung, klicken Sie hier um dies zu
  verhindern.}
\item
  \emph{Sie haben neulich eine Überweisung von 300 Euro vorgenommen, klicken
  Sie hier um dies rückgängig zu machen.}
\item
  \emph{Wie gewünscht finden sie anbei das Dokument über unser Gespräch.}
\end{itemize}

\subsubsection{E-Mail-Adressen Verschleierung}
\label{ssub:e-mail-adressen-verschleierung}

Manchmal müssen E-Mail-Adressen auf Webseiten aufscheinen. Sei es auf der
Kontakt-Seite, in den Lizenzvereinbarungen oder im Impressum. Über die Jahre
haben sich verschiedenste Techniken entwickelt, E-Mail-Adressen auf Webseiten
zu verschleiern. Im Folgenden werden drei Methoden beschrieben, die
E-Mail-Adressen für Menschen lesbar und für Software-Agenten unlesbar machen.

\paragraph{Interpunktion verschleiern}
\label{par:interpunktion-verschleiern}

Das Umschreiben einer Adresse vom Format \enquote{name@example.org} in das
Format \enquote{name {[}at{]} example {[}dot{]} org} verändert das Muster, nach
dem Software-Agenten suchen müssen. Zusätzlich lassen sich viele weitere
Variationen desselben Tricks erdenken: \enquote{name-AT-example-DOT-org}.
Obwohl es für Bots nicht unmöglich ist diese Verschleierung zu durchschauen,
ist sie immer noch weitaus besser, als die E-Mail-Adresse ohne Verschleierung
zu veröffentlichen.\footcite{obfuscateEmailAddresses}

\paragraph{CSS/JavaScript}
\label{par:cssjavascript}

Beinahe alle Bots lesen lediglich HTML-Dokumente und ignorieren dabei externe
Ressourcen wie CSS und JavaScript.\footcite{hideEmailAddress} Der zusätzliche
Aufwand, diese externen Ressourcen anzufragen, das HTML-Dokument in ein
logisches Modell (DOM) umzuwandeln und folglich den CSS- und JavaScript-Code
auf dieses Modell anzuwenden, ist verhältnismäßig groß und raubt Zeit, die für
die Suche anderer Adressen genutzt werden könnte. Ein mit CSS verstecktes HTML
Element, das inmitten einer E-Mail-Adresse platziert wird, ist eine nahezu
perfekte Verschleierungs-Technik.\footcite{obfuscateEmailAddresses}

\begin{lstlisting}[language=HTML]
<style>
[data-block~=bots] { display: none }
</style>
<p>name@<span data-block="bots">nil</span>example.org</p>
\end{lstlisting}

Mit JavaScript lassen sich E-Mail-Adressen durch eine Codierung oder
Verschlüsslung vor Bots verbergen. Diese Methoden bieten einen für Bots
praktisch unknackbaren Schutz. Während CSS bei so gut wie 100\% aller
menschlichen Benutzer aktiviert ist,\footnote{CSS ist für die Gestaltung von
Webseiten zuständig und bietet daher nicht eine Vielzahl an Sicherheitslücken,
für welche einige Nutzer JavaScript deaktivieren. Die Anzahl an
Webseiten-Benutzern, welche CSS deaktiviert haben, ist derart gering, dass sich
zu diesem Thema keine fundierten Statistiken finden lassen.} ist JavaScript nur
bei 98\,\% aktiviert.\footcite{javaScriptDisableStats} JavaScript basierte
Verschleierungen grenzen damit 2\,\% aller Benutzer aus, erlauben jedoch eine
beliebige Verschlüsslung der Adresse, darunter zum Beispiel
ROT13.\footcite{modernCryptanalysis}

\paragraph{HTML-Codierung}
\label{par:html-codierung}

HTML unterstützt selbst eine Reihe von Codierungen,\footnote{Ein Generator für
codierte E-Mail-Adressen findet sich unter
\url{http://rumkin.com/tools/mailto_encoder/custom.php}} die auf
E-Mail-Adressen angewandt werden kann.

\subparagraph{URL-Codierung}
\label{spar:url-codierung}

Bei einer URL Codierung werden einzelne Zeichen durch ein Prozentzeichen
gefolgt von zwei alphanumerischen Zeichen ersetzt. Die Kombination der beiden
alphanumerischen Zeichen lässt sich in standardisierten Tabellen
nachschlagen.\footcite{rfc:3986}

\begin{lstlisting}[language=HTML]
<a href="mailto:%6e%61%6d%65%40%65%78%61%6d%70%6c%65%2e%6f%72%67">
  name@example.org
</a>
\end{lstlisting}

\subparagraph{HTML-Entitäten-Referenzen}
\label{spar:html-entitaeten-referenzen}

Eine Zeichen-Entität-Referenz beschreibt ein Zeichen, indem der Code-Point link
von \texttt{\&\#} oder \texttt{\&\#x} und rechts von \texttt{;} umgeben wird.
Der Code-Point beschreibt die numerische Position, an welcher das Zeichen im
Unicode-Standard geführt wird. Beginnt eine Entitäten-Referenz mit
\texttt{\&\#x}, muss der Code-Point in Hexadezimalzahlen, anstelle von
Dezimalzahlen, angegeben werden.

\begin{lstlisting}[language=HTML,escapechar=\$]
<a href="mailto:&#110;&#97;&#109;&#101;&#64;&#101;&#120;&#97;$\ldots$">
  name@example.org
</a>
\end{lstlisting}

\subparagraph{URL-Codierung + HTML-Entitäten-Referenzen}
\label{spar:url-codierung-html-entitaeten-referenzen}

URL-Codierung kann auch mit HTML-Entitäten-Referenzen kombiniert werden.

\begin{lstlisting}[language=HTML,escapechar=\$]
<a href="mailto:%6e%61&#109;&#101;&#64;&#101;&#120;&#97;&#109;$\ldots$">
  name@example.org
</a>
\end{lstlisting}

\subsubsection{Kommentare}
\label{ssub:kommentare}

Neben E-Mails stellen online-Kommentare ein weiteres Medium für
Spam-Nachrichten dar. Um Kommentare vor Bots zu schützen, lassen sich
verschiedene Techniken zum Einsatz bringen. Dazu zählen
CAPTCHAs\footref{ssub:captcha}, Honeypots,\footref{ssub:honeypot} die Pflicht,
sich beim Kommentarsystem anzumelden und Zwei-Schritt-Absenden-Verfahren.

Im Folgenden wird eine Spam-Attacke beschrieben, die auf ein Kommentarsystem
vom 6. Mai 2015 (20:57 Uhr) bis zum 7. Mai 2015 (06:21 Uhr) durchgeführt wurde.
Das Kommentarsystem befand sich unter der Adresse
\texttt{*.blogfill.de/weblog/funktionsweise-von-easing-funktionen} und setzte
als einzige Anti-Spam-Technik eine Kommentarvorschau ein, die vor dem Absenden
eines Kommentars durchgeführt werden musste. Das alleine bot bereits einen
guten Schutz gegen Spam,\footcite{howSkipCommentPreview} verhinderte jedoch
nicht, dass im genannten Zeitraum 719 Spam-Kommentare eingereicht wurden.

Bei den 719 Kommentaren wurde der Name \enquote{Walter} siebenmal und damit am
häufigsten gewählt. Es wurden insgesamt 71 Links publiziert, dabei handelt es
sich um 35 voneinander unterschiedliche Exemplare. Die drei häufigsten Links
wurden jeweils viermal publiziert, einer davon verwendete HTTPS im Gegensatz zu
HTTP. Aus den 35 unterschiedlichen Links verwendeten nur zwei HTTPS, beide mit
demselben Hostname: \texttt{ummgc.org}. Die Verwendung von HTTPS ist insofern
interessant, als dass HTTPS ein Zertifikat verlangt, welches von einer zumeist
externen Zertifizierungsstelle ausgestellt werden muss. Spammer können so
leicht rückverfolgt werden.

Auch interessant ist es, dass von insgesamt 376.165 Zeichen 0,053\,\%
Steuerzeichen sind, also nicht normale Zeichen, die am Bildschirm dargestellt
werden, sondern die Darstellung anderer Zeichen beeinflussen. Solche seltene
Zeichen könnten eingesetzt werden, um in primitiven Systemen, die nicht mit
solcherlei Zeichen rechnen, einen Fehler zu verursachen. 0,27\,\% aller Zeichen
machen Währungszeichen aus, davon 914 Euro-, 91 Dollar- und 22 Pfund-Zeichen.

Ein vollständiger Report aller Daten und eine Beschreibung, wie sie gewonnen
wurden, findet sich online unter
\url{https://github.com/SoftwareAgenten/GA-Archive/tree/master/Comments}.
