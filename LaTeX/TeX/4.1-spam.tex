\subsection{Spam} % (fold)
\label{sub:spam}
\subsubsection{Definition} % (fold)
\label{ssub:spam_definition}
Unerwünschte Nachrichten im Internet, größtenteils Werbeangebote,
Phishing-Attacken oder Übermittler von Schadsoftware, werden als Spam
bezeichnet. Spam-Nachrichten stellen per definitionem eine leidige
Kommunikationsform dar. Die gewöhnlichste Art, namhaft durch ihre Omnipräsenz
im Leben mit dem Internet, stellt der E-Mail-Spam dar.

Der Begriff fand noch vor dem Erscheinen des \emph{World Wide Web} anklang,
beschrieb damals jedoch eine Flut an unerwünschten
Nachrichten.\footcite{originTermSpam} Erste Aufkommen des Phänomens wurden im
ARPANET\footcite{junkMailProblem} und im USENET beobachtet.

Der Name \emph{Spam} entspringt dem gleichnamigen Dosenfleisch, welches im
Zweiten Weltkrieg in großen Mengen an Soldaten verteilt
wurde.\footcite{lifeDuringSecondWW} Der Markenname Spam® (kurz für \emph{Spiced
Ham}) setzte sich auf diese Weise rasch im Vereinigten Königreich als Deonym
für Frühstücksfleisch durch. Die britische Comedy-Gruppe Monty Python griff im
Sketch \emph{Spam}\footcite{spamMontyPython} der BBC Serie \emph{Monty Python’s
Flying Circus} die Allgegenwärtigkeit des Fleischprodukts auf, weshalb dieses
bis heute als Symbol für einen unerwünschten Überschuss fungiert.

Mit dem Begriff \emph{Spam} werden heutzutage diverse Arten von Nachrichten
bezeichnet. Werbenachrichten sind das klassische Beispiel einer Spam-Nachricht.
Sie bieten zumeist dubiose Produkte oder Dienstleistungen an.
% subsubsection spam_definition (end)
% subsection spam (end)