\subsection{Software-Agenten} % (fold)
\label{sub:software_agenten}
\subsection{Definition} % (fold)
\label{sub:software-agent_definition}
Ein Software-Agent ist eine Computer-Applikation, welche autonom vordefinierte
Ziele verfolgt. Alternative Namen umfassen: \emph{Bot}, \emph{autonome
Software}, oder im Bereich der Künstlichen Intelligenz auch \emph{KI} oder
\emph{A.I}.

Es gibt verschiedene Arten von Software-Agenten die verschiedenste Aufgaben
übernehmen. Diese Arbeit befasst sich mit Bots, die das Internet als
\emph{Lebensraum} nutzen. Dazu zählen unter anderen:

\begin{description}
  \item[Webcrawler/Spider]
  Bots, welche Webseiten im \emph{World Wide Web} indexieren und/oder
  analysieren. Die bekanntesten Vertreter sind \emph{Web
  Spiders}\footcite{tkWWWRobot}: Agenten, welche von Suchmaschinen eingesetzt
  werden um neue Webseiten zu finden und den bestehenden Suchindex zu
  aktualisieren. Webcrawler können auch genutzt werden um E-Mail-Adressen
  aufzuspüren, beispielsweise für den Versand von Spam-Nachrichten.
  
  \item[Spambot]
  Diese Software-Agenten versenden Spam-Nachrichten. Da ihre Spam-Nachrichten
  jedoch meist auf sie zurückverfolgt werden können (zum Beispiel mittels
  Absender-Adresse oder Benutzername) sehen sich Spambots genötigt massenhaft
  Accounts anzulegen um nicht blockiert zu werden. Im Kapitel
  \ref{ssub:captcha} wird eine Methode vorgestellt, um dies zu vermeiden und in
  \ref{sub:maschinelles_lernen_captcha} wird ein Versuch beschrieben diese
  Methode zu überlisten.
  
  \item[Brute-Force-Bots]
  Brute-Force (englisch für \emph{rohe Gewalt}) beschreibt den Akt, bei welchem
  eine Software sämtliche möglichen Optionen erproben um in fremdes System
  einzudringen. In der Regel besteht dies darin Accounts zu entwenden, indem
  eine Kombination von Benutzernamen und Passwörtern ausprobiert werden. Der
  Benutzername kann beispielsweise in Form einer E-Mail-Adresse bezogen worden
  sein, welche ein Webcrawler auftreiben hat. Die Passwörter, die erprobt
  werden, entstammen einer Tabelle von geläufigen Passwörtern.
\end{description}
% subsection software-agent_definition (end)
% subsection software_agenten (end)