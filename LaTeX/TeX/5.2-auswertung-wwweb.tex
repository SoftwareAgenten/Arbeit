\subsection{Auswertung der WWWEB-Daten}
\label{sub:auswertung-der-wwweb-daten}

Am 4. Juli 2016 sammelte WWWEB Daten von 2190 verschiedenen Internet-Domänen
über die Zeitspanne von 18:43:25 Uhr bis um 21:43:19 Uhr. Das entspricht zwölf
Domänen pro Minute. Als Startdomäne wurde \texttt{addpixel.net} gewählt. WWWEB
navigierte mit dem Standard User-Agent von \texttt{axios/0.12.0}.

Aus den dabei gescannten robots.txt-Dateien können verschiedenste Informationen
gewonnen werden.\footnote{Der vollständige Bericht findet sich unter
\url{https://github.com/SoftwareAgenten/GA-Archive/blob/master/WWWEB/wwweb-repor
t_2016-06-04_18-43-25.json}} Zur Analyse des Berichts wurde die
\emph{GA-Analyzer} Software\footnote{Projekt-Webseite:
\url{https://github.com/SoftwareAgenten/GA-Analyzer}} eingesetzt, da sie sich
in eine interaktive Node.js Umgebung laden lässt und dort dedizierte Funktionen
für das Untersuchen von JSON-Strukturen bereitstellt.

\subsubsection{Nicht zugelassene Verzeichnisse}
\label{ssub:nicht-zugelassene-verzeichnisse}

\begin{tabular}{ l|r }
  \textbf{Pfad} & \textbf{Anzahl}\\
  \hline
  \texttt{/wp-admin/} & 462\\
                      & 201\\
  \texttt{/search/}   & 256\\
  \texttt{/admin/}    & 189\\
  \texttt{/includes/} & 188\\
  \texttt{/modules/}  & 163\\
  \texttt{/scripts/}  & 158\\
  \texttt{/wp-admin/} & 158\\
  \texttt{/search/}   & 462\\
\end{tabular}

\emph{Die acht am häufigsten verbotenen Pfade und deren Anzahl}

In TK\{Tabelle \#\} zeigt sich unter anderem die Dominanz von WordPress
gegenüber anderen Content-Management-Systemen; der Pfad \texttt{wp-admin}
beschreibt die Lage der Administratoren-Oberfläche von WordPress. Auf Platz
eins und drei finden sich zwei gleichgültige Ausdrücke, beide verbieten einem
User-Agenten sämtliche Unterverzeichnisse sowie das Wurzelverzeichnis
anzufragen. Auf Platz vier findet sich ein typischer Pfad zu einer Suchseite.
Dieser wird oft von Software-Agenten verborgen, um eine bessere Platzierung bei
Suchmaschinen zu erreichen.\footnote{TK
\url{https://www.mattcutts.com/blog/search-results-in-search-results/}
2016-06-05}

\texttt{includes}, \texttt{modules} und \texttt{scripts} hingegen sind typische
Verzeichnisse, in welchen Bausteine einer Webseite platziert werden. Diese
Bausteine alleine bieten weder Bots noch Menschen einen Mehrwert und werden
daher als verboten markiert. Die vollständige Liste aller von WWWEB indexierten
Verzeichnisse findet sich im WWWEB-Archiv.\footnote{Zu finden unter:
\url{https://github.com/SoftwareAgenten/GA-Archive/tree/master/WWWEB}}

\subsubsection{User-Agents}
\label{ssub:user-agents}

\begin{tabular}{ l|r }
  \textbf{User-Agent} & \textbf{Anzahl}\\
  \hline
  *         & 2501\\
  Googlebot & 124\\
  Slurb     & 110\\
  008       & 109\\
  AhrefsBot & 83\\
\end{tabular}

\emph{Die vier meist gelisteten User-Agenten und deren Anzahl}

TK\{Tabelle \#\} zeigt, dass der am häufigsten gelistete User-Agent *
(Asterisk) ist. Dieser Asterisk wird als \emph{Wildcard} bezeichnet und steht
dabei für sämtliche User-Agenten. Auf den Plätzen zwei und drei finden sich
\emph{Googlebot} und \emph{Slurb}, zwei Webspider{[}TK REF SPIDER{]} die
jeweils der Google-Suche und der Yahoo!-Suche zugehörig sind. \emph{008} ist
der User-Agent des \emph{80legs}-Webcrawlers, welcher als kommerzielle
Dienstleistung angeboten wird. \emph{AhrefBot} scannt das Web nach Backlinks
und anderen Informationen und stellt diese Daten in Form einer kommerzielle
Dienstleistung bereit. Die vollständige Tabelle aller User-Agenten und deren
Anzahl findet sich ebenfalls im WWWEB-Archiv.\footnote{Zu finden unter:
\url{https://github.com/SoftwareAgenten/GA-Archive/tree/master/WWWEB}}
