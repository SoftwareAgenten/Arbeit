\subsection{Künstliche Intelligenz} % (fold)
\label{sub:kunstliche_intelligenz}
\subsubsection{CAPTCHA} % (fold)
\label{ssub:captcha}

% subsubsection captcha (end)

\subsubsection{Honeypot} % (fold)
\label{ssub:honeypot}
Als Honeypot (zu Deutsch \emph{Honigtopf}, sinngemäß \emph{Fettnäpfchen}) wird in der Informatik eine Falle bezeichnet, die sich als begehrenswertes Objekt tarnt.

Ein exemplarischer Einsatz ist das von Ordnungshütern durchgeführte Veröffentlichen von scheinbar illegalen Dateien im Internet.
Nutzer, welche auf diese Dateien zugreifen, werden vom Honeypot registriert und können daraufhin strafrechtlich verfolgt werden.

Dieselbe Methodik lässt sich bei der Unterscheidung zwischen Bots und Menschen zum Einsatz bringen.
Eine geläufige Praxis ist es, online Formulare (zum Beispiel ein Kommentar-Formular) mit einem zusätzlichen Textfeld zu bestücken und dieses visuell zu verbergen.
Ein menschlicher Benutzer kann dieses Feld weder erreichen noch wahrnehmen; konträr stoßen Software-Agenten auf das Textfeld – sie nehmen schließlich nur den Programmcode des Formulars und nicht die visuelle Aufmachung wahr – und füllen es aus.
Beim Absenden des Formulars kann die Formular-Software überprüfen, ob das Honeypot-Feld ausgefüllt wurde oder nicht; im ersten Falle würde diese Software das Absenden unterbinden.
% subsubsection honeypot (end)
% subsection kunstliche_intelligenz (end)