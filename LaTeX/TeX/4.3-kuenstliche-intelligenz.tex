\subsection{Künstliche Intelligenz}
\label{sub:kunstliche_intelligenz}

Die ersten Software-Bots waren noch weit entfernt vom selbstständigen Verhalten
der heutigen Software-Agenten. Jeder Bot wurde für eine einzige Aufgabe
programmiert,\footcite{wa:spamFilter} beispielsweise Spam-Emails an eine
vordefinierte Liste von Empfängern zu senden. Diese Verfahrensweise skaliert
nicht in die Dimensionen, die das Internet inzwischen angenommen hat.
Gleichzeitig mit dem Internet sind Computer immer schneller, komplexer und
leistungsfähiger geworden. So lassen sich heute autonome Anwendungen
realisieren, die selbst auf der Suche nach E-Mail-Adressen das WWW
durchstöbern, zur Situation passende Texte verfassen, diese versenden und auf
Antworten warten.

Das Motto eines voll-autonomen Software-Agenten ist es, den menschlichen Faktor
bei der Beschaffung von Daten zu eliminieren. Wie oben beschrieben, sind Bots
besonders dazu geeignet, Listen von Daten anzulegen und diese abzuarbeiten.
Manchmal wird das von Benutzern des Internets geduldet -- wie bei der
Indexierung von öffentlichen Webseiten -- und manchmal nicht -- wie bei der
Aufspürung von privaten Daten.

Um Bots von gewissen Orten im Internet auszusperren, wurden verschiedene
Techniken entwickelt, darunter CAPTCHAs,\footref{ssub:captcha}
Honeypots,\footref{ssub:honeypot} und der Zwang zur Beglaubigung als Mensch.
Beim Zwang zur Beglaubigung als Mensch, handelt es sich um das Vorweisen von
persönlichen Daten, die den Benutzer als Menschen authentifizieren. Dazu zählen:

\begin{itemize}
\item
  Ein Nachrichten-Dienst, der seine Nachrichten über das Internet versendet,
  aber dennoch die Handynummer bei der Registrierung eines neuen Accounts
  verlangt. Die Handynummer dient in diesem Beispiel als rares Gut, das ein Bot
  nicht beliebig beschaffen kann.
\item
  Ein Online-Shop, der bereits vor der ersten Abbuchung nach
  Kreditkarteninformationen fragt.
\item
  Eine Online-Gemeinschaft, die beim Beitritt eines neuen Mitglieds eine
  digitale Kopie des Personalausweises verlangt.
\end{itemize}

\subsubsection{CAPTCHA}
\label{ssub:captcha}

Eine aktive Software-Bot Blockade wird als CAPTCHA (für \emph{Completely
Automated Public Turing Test To Tell Computers and Humans
Apart}\footcite{captchaNet}, zu Deutsch: \emph{Vollständig automatisierter,
öffentlicher Turing-Test zur Unterscheidung von Computern und Menschen})
bezeichnet. Exemplarische Einsatzgebiete sind die Reduzierung von
Spam-Nachrichten durch ein Online-Kontaktformular und der Ausschluss von Bots
bezüglich der Registrierung bei einem sozialen Netzwerk.

Verbreitete Ausführungen der CAPTCHA-Methode umfassen:

\begin{description}
  \item[Verzerrter Text]
  Der Benutzer muss eine verzerrte Abbildung einer Zeichenfolge in ein Textfeld
  eingeben. Die Verzerrung ist stark genug, so dass sie im Idealfall nicht von
  Software ausgemacht werden kann und dabei dennoch für Menschen lesbar bleibt.
  
  \item[Objekt-Klassifikation]
  Dem Benutzer wird eine Gruppe von Bildern präsentiert, von welchen jene
  markiert werden müssen, die zu einer bereitgestellten Beschreibung passen.
  
  \item[Akustisches Diktat]
  Verschiedene Stimmen diktieren dem Benutzer einen alphanumerischen Code der
  in ein Textfeld eingegeben werden muss. Diese Methode ergänzt zumeist andere
  Ausführungen der CAPTCHA-Methode um Rücksicht auf visuell eingeschränkte
  Benutzer zu nehmen.
\end{description}

\subsubsection{Honeypot}
\label{ssub:honeypot}

Als Honeypot (zu Deutsch \emph{Honigtopf}) wird in der Informatik eine Falle
bezeichnet, die sich als begehrenswertes Objekt tarnt.

Ein exemplarischer Einsatz ist das von Ordnungshütern durchgeführte
Veröffentlichen von scheinbar illegalen Dateien im Internet. Nutzer, welche auf
diese Dateien zugreifen, werden vom Honeypot registriert und können daraufhin
strafrechtlich verfolgt werden.

Dieselbe Methodik lässt sich bei der Unterscheidung zwischen Bots und Menschen
zum Einsatz bringen. Eine geläufige Praxis ist es, Onlineformulare (zum
Beispiel ein Kommentar-Formular) mit einem zusätzlichen Textfeld zu bestücken
und dieses visuell zu verbergen. Ein menschlicher Benutzer kann dieses Feld
weder erreichen noch wahrnehmen; konträr stoßen Software-Agenten auf das
Textfeld -- sie nehmen schließlich nur den Programmcode des Formulars und nicht
die visuelle Aufmachung wahr -- und füllen es aus. Beim Absenden des Formulars
kann die Formular-Software überprüfen, ob das Honeypot-Feld ausgefüllt wurde
oder nicht; im ersten Falle würde diese Software das Absenden unterbinden.

\subsubsection{Prognose}
\label{ssub:prognose}

Die Anzahl an Bots, die das Internet durchstreifen, wird auch in den nächsten
Jahren zunehmen. Dabei wird sich die Kluft zwischen einfach gestrickten Agenten
und Künstlicher Intelligenz immer weiter vergrößern. Einfache Bots,
beispielsweise Beschaffer von E-Mail-Adressen, versuchen so große Teile des
Internets so gut wie möglich zu durchsuchen. Dabei muss der Algorithmus
leistungsstark, energiesparend und effizient sein. Daher werden diese Bots auch
in 50 Jahren noch auf Honeypots hereinfallen, dafür aber immer mehr Webseiten
durchsuchen, um diese Schwäche wettzumachen. Am anderen Ende des Spektrums
finden sich Agenten, die von einer Künstlichen Intelligenz angetrieben werden.
KI gesteuerte Bots werden immer wichtigere Aufgaben übernehmen. Dies trifft
sowohl für gut- als auch bösartige Bots zu. Damit CAPTCHAs in einigen Jahren
noch gegen KIs standhalten können, müssten sie derart kompliziert werden, dass
auch die meisten Menschen nichts mehr damit anzufangen wüssten. Dieser Trend
ist bereits heute zu beobachten und führt zum \emph{Zwang zur Beglaubigung als
Mensch}. Immer mehr Online-Plattformen verlangen nach einer Telefonnummer um
sich Bots vom Leibe zu halten. Anderseits wandern immer mehr persönliche Daten
vom Web in Apps; gerade soziale Netzwerke lösen sich vom WWW zugunsten mobiler
Applikationen.\footcite{timeOnSmartphone} Diese Fragmentierung lukrativer Daten
macht sie für Bots schwieriger zugänglich.
