\subsection{Künstliche Intelligenz} % (fold)
\label{sub:kunstliche_intelligenz}
\subsubsection{CAPTCHA} % (fold)
\label{ssub:captcha}
Eine aktive Software-Bot Blockade wird als CAPTCHA (für \emph{Completely
Automated Public Turing Test To Tell Computers and Humans
Apart}\footcite{captchaNet}, zu Deutsch: \emph{Vollständig
automatisierter, öffentlicher Turing-Test zur Unterscheidung von
Computern und Menschen}) bezeichnet. Exemplarische Einsatzgebiete sind
die Reduzierung von Spam-Nachrichten durch ein online Kontaktformular
und der Ausschluss von Bots bezüglich der Registrierung bei einem
sozialen Netzwerk.

Verbreitete Ausführungen der CAPTCHA-Methode, welche für diese Arbeit
von Wichtigkeit sind, umfassen:

\begin{description}
  \item[Verzerrter Text]
  Der Benutzer muss eine verzerrte Abbildung einer Zeichenfolge in ein
  Textfeld eingeben. Die Verzerrung ist stark genug, als dass sie im
  Idealfall nicht von Software ausgemacht werden kann und dabei dennoch
  für Menschen lesbar bleibt.
  
  \item[Objekt Klassifikation]
  Dem Benutzer wird eine Gruppe von Bildern präsentiert, von welchen jene
  markiert werden müssen, welche zu einer bereitgestellten Beschreibung
  passen.
  
  \item[Akustisches Diktat]
  Verschiedene Stimmen diktieren dem Benutzer einen alphanumerischen Code
  der in ein Textfeld eingegeben werden muss. Diese Methode ergänzt zumeist
  andere Ausführungen der CAPTCHA-Methode um Rücksicht auf visuell
eingeschränkte Benutzer zu nehmen.
\end{description}
% subsubsection captcha (end)

\subsubsection{Honeypot} % (fold)
\label{ssub:honeypot}
Als Honeypot (zu Deutsch \emph{Honigtopf}, sinngemäß \emph{Fettnäpfchen}) wird in der Informatik eine Falle bezeichnet, die sich als begehrenswertes Objekt tarnt.

Ein exemplarischer Einsatz ist das von Ordnungshütern durchgeführte Veröffentlichen von scheinbar illegalen Dateien im Internet.
Nutzer, welche auf diese Dateien zugreifen, werden vom Honeypot registriert und können daraufhin strafrechtlich verfolgt werden.

Dieselbe Methodik lässt sich bei der Unterscheidung zwischen Bots und Menschen zum Einsatz bringen.
Eine geläufige Praxis ist es, online Formulare (zum Beispiel ein Kommentar-Formular) mit einem zusätzlichen Textfeld zu bestücken und dieses visuell zu verbergen.
Ein menschlicher Benutzer kann dieses Feld weder erreichen noch wahrnehmen; konträr stoßen Software-Agenten auf das Textfeld – sie nehmen schließlich nur den Programmcode des Formulars und nicht die visuelle Aufmachung wahr – und füllen es aus.
Beim Absenden des Formulars kann die Formular-Software überprüfen, ob das Honeypot-Feld ausgefüllt wurde oder nicht; im ersten Falle würde diese Software das Absenden unterbinden.
% subsubsection honeypot (end)
% subsection kunstliche_intelligenz (end)