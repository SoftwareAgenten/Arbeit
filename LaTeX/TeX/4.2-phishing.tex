\subsection{Phishing} % (fold)
\label{sub:phishing}
\subsubsection{Definition} % (fold)
\label{ssub:definition}
Das Bestreben persönliche Daten im Internet durch ein gezielt verfälschtes Auftreten abzugreifen wird als Phishing (aus dem Englischen \emph{fishing}, für \emph{fischen} oder \emph{angeln}) bezeichnet.
In der Regel handelt es sich dabei um E-Mail-Nachrichten oder Webseiten, welche Informationen wie Name, Adresse, Benutzername/Passwort oder Kreditkarten-Daten anfordern.
Dem Empfänger wird dargelegt, sein Account auf einer Webseite würde in Gefahr schweben, sollten die geforderten Informationen nicht rasch übergeben werden.
Dieser künstliche Drang soll den Adressaten der Phishing-Nachricht davon abhalten seinen Verstand zu benutzen und stattdessen eine panische, reflexartige Handlung auslösen.
Diese Handlung wird vom Betrüger geschickt dirigiert: dem Opfer wird ein großer Button oder ein einfach erreichbarer Link präsentiert, der auf die Webseite des Betrügers führt.
Diese Webseite imitiert das Aussehen des Originals, leitet jedoch die vom Benutzer eingetragenen Daten an den Sender der Phishing-Nachricht weiter.

Phishing ist eine besonders elegante Art des Spams, da die Opfer häufig nicht mitbekommen, dass ihre Daten von Dritten abgegriffen werden.
Nachdem man die geforderten Daten eingegeben und abgesendet hat, leitet die Webseite des Betrügers das Opfer auf die tatsächliche Webseite weiter.
% subsubsection definition (end)
% subsection phishing (end)
