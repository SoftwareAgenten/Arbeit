\label{topic:great-attractor}
Die Daten über das Browse-Verhalten von Software-Agenten wurden von einer
eigens für diese Arbeit erstellten Software-Suite gesammelt. Die Suite wurde
unter dem Namen \emph{Great Attractor}\footnote{Der Name wurde in Anlehnung an
das kosmische Filament, im deutschsprachigen Raum auch als \emph{Großer
Attraktor} bekannt, gewählt.} im Internet unter der MIT-Lizenz\footnote{Eine
Kopie der Lizenz ist unter \url{https://legal.addpixel.net/licenses/MIT/} zu
finden.} auf der Open-Source Code Plattform GitHub.com
veröffentlicht.\footnote{GitHub.com Projektseite:
\url{https://github.com/SoftwareAgenten/Great-Attractor}}

Die MIT-Lizenz erlaubt die Vervielfältigung, Veränderung und Unterlizenzierung
der \emph{Great Attractor} Softwares sowohl für den privaten als auch den
kommerziellen Gebrauch. Dabei muss die Lizenzangabe samt Copyright Anmerkung
unverändert beibehalten werden.\footnote{Hierbei handelt es sich lediglich um
eine kurze Zusammenfassung die keinerlei Anspruch auf Vollständigkeit oder
Rechtsgültigkeit ausübt. Eine vollständige Kopie der Lizenz findet sich unter
\url{https://legal.addpixel.net/licenses/MIT/}.} Zufolge der Open Source
Initiative ist die MIT-Lizenz der Open Source Definition
konform.\footcite{license:MIT} \emph{Open Source} bedeutet in diesem Kontext:
der Quellcode der \emph{Great Attractor} Softwares kann frei eingesehen werden
und in anderen Software-Projekten zum Einsatz kommen.

GitHub.com wurde zugrunde der folgenden Argumente als Open-Source Code
Plattform gewählt.

\begin{enumerate}
\item
  GitHub.com ist zur Zeit dieses Schreibens die populärste Open-Source Code
  Plattform\footcite{githubPop} und erfreut sich demgemäß einer großen Anzahl an
  aktiven Nutzern.
\item
  GitHub setzt auf der \emph{git} Versionsverwaltungssoftware auf, welche sowohl
  bei den \emph{Great Attractor} Softwares zum Einsatz kommt als auch bei der
  Erstellung der \emph{Software-Agenten im Internet} Arbeit verwendet wird.
\end{enumerate}

\subsection{Architektur und Philosophie}
\label{sub:architektur_und_philosophie}
\emph{Great Attractor} (GA) besteht aus zwei Kernkomponenten: dem gleichnamigen
Archiv an von Bots produzierten Daten\footnote{vollständig einsichtbar unter
\url{https://github.com/SoftwareAgenten/Great-Attractor}} und eine Reihe an
Applikationen, die Datensätze für das Archiv sammeln.\footnote{Eine Liste alle
Applikation kann unter
\url{https://github.com/SoftwareAgenten/Great-Attractor\#data-collectors}
eingesehen werden.}

Das GA-System erfüllt folgende Anforderungen.

\begin{itemize}
\item
  \textbf{Portabilität der Daten}. Sämtliche durch GA gesammelte Daten werden im
  JSON-Format auf dem Dateisystem gesichert. Dies erleichtert es
  Sicherungskopien der Datensätze anzufertigen und diese folglich zu
  veröffentlichen bzw. auf anderen Wegen zu vervielfältigen.
\item
  \textbf{Portabilität des Systems}. GA wurde in der Programmiersprache PHP
  verfasst, welche auf allen gängigen Computer-Betriebsystemen\footnote{
  Darunter: \emph{Linux}, \emph{FreeBSD}, \emph{Unix}, \emph{Windows}
  und \emph{Darwin}} genutzt werden kann. Außerdem ist das JSON
  Datenaustauschformat, in welchem Daten gesichert werden, einfach für Menschen
  zu lesen\footcite{jsonDe} und für Maschinen einfach zu
  parsen.\footcite{jsonDe} Dies erlaubt es nicht-GA-Systemen auf die gewonnenen
  Daten zurückzugreifen und ergänzende Operationen auszuführen.
\end{itemize}
