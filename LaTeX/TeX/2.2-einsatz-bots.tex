\subsection{Einsatz von Software-Agenten}
\label{sub:einsatz-von-software-agenten}

Das Internet ist inzwischen zu groß, um allein von Menschen kuriert und
gepflegt zu werden. So haben sich immer mehr Arten von Bots gebildet, die immer
spezifischere Aufgaben übernehmen. Die Einsatzgebiete lassen sich grob in zwei
Lager aufteilen: gutartige Bots und bösartige Bots. Gutartige Bots zeichnen
sich dadurch aus, dass sie gemeinnützige, konstruktive Arbeiten leisten und
sich bei Bedarf auch freiwillig blockieren lassen. Folgende Listen führen die
wichtigsten Vertreter auf.

\subsubsection{Gutartige Bots}
\label{ssub:gutartige-bots}

\begin{description}
  \item[Search Robots]
  Ein Paradebeispiel für gutartige Bots sind Webcrawler, die sich auf die
  Indexierung von Webseiten für Suchmaschinen spezialisiert haben. Das W3C, das
  Standardisierungsgremium für das \emph{World Wide Web},
  \footcite{w3c:standards} definiert mehrere Standards, die es
  Website-Administratoren erlauben, Bots den Zugang zur Webseite vollständig,
  oder auch nur zu Teilen, zu unterbinden.\footcite{w3c:searchBots}
  \label{robotstxt} Der am weitesten verbreitete Standard spezifiziert eine
  Datei namens \texttt{robots.txt}, welche Blockaden für Bots
  definiert.\footcite{w3c:robotsTxt} Alle großen Suchmaschinen erklären ihre
  Konformität zu einem oder mehreren dieser
  Standards.\footcite{robotsTagGoogle}\footcite{yandexUsingRobotsTxt}\footcite{duckduckgoBot}
  
  \item[Feed-Crawler]
  Damit RSS-Clients stets mit den neuesten Inhalten gefüttert werden können,
  müssen Bots RSS-/Atom-Feeds periodisch anfragen. Diese Feeds werden von
  Weblogs und anderen Webseiten zumeist exklusive für Feed-Crawler angelegt und
  bereitgestellt.
  
  \item[Webseite-Monitoring]
  Um sicherzustellen, dass die eigene Webseite erreichbar ist, lassen sich
  Software-Agenten einsetzen, welche die Webseite im Minutentakt anfragen und
  bei einem eventuellen Ausfall Alarm schlagen. Dieser Alarm kann sich
  beispielsweise in Form einer E-Mail-Nachricht an den Administrator äußern.
  
  \item[Schutzengel]
  Da viele bösartige Bots Links in sozialen Netzwerken teilen, überprüfen
  einige Netzwerke die verlinkte Seite, bevor der Link publiziert wird. Der
  Mikroblogging-Dienst Twitter geht in diesem Bereich beispielhaft voran.
  Sämtliche auf Twitter publizierten Links werden vom hauseigenen
  URL-Kürzungsdienst \emph{t.co} verpackt.\footcite{twitterShortLink}
\end{description}

\subsubsection{Bösartige Bots}
\label{ssub:boesartige-bots}

\begin{description}
  \item[Bösartige Webcrawler]
  Konträr zu \emph{freundlichen} Webcrawlern, wie Search Bots, durchsuchen
  bösartige Webcrawler das Web nach persönlichen Daten, angreifbaren
  Anmelde-Seiten oder ungeschützten Formularen. Ironischerweise begutachten
  diese Bots ebenfalls die \texttt{robots.txt}-Dateien, um eben jene Bereiche
  ausfindig zu machen, welche vom Administrator als Tabus markiert wurden.
  
  \item[Spam-Bots]
  Diese Bots versenden Spam an Adressen, die Webcrawler aufgetrieben haben.
  Dabei muss es sich nicht zwangsweise um E-Mail-Spam handeln; auch
  Kommentarsysteme, Online-Foren, soziale Netzwerke\footcite{facebookSpamBiz}
  und Instant-Messaging-Dienste wie WhatsApp\footcite{whatsappSpamSperre} sind
  betroffen.
  
  \item[Statistiken-Verfälschung]
  Mit der Kommerzialisierung des Internets wurden Statistiken immer wichtiger.
  Werbetreibende zahlen pro Ad Impression,\footcite{wp:adImpression} also pro
  Aufruf. Botnets, also Zusammenschlüsse vieler Bots, können genutzt werden,
  zusätzlichen Verkehr auf diese Werbeanzeigen zu
  leiten.\footcite{fakeTrafficPayday} Da Bots oft weniger kosten als die
  Werbeanzeigen einbringen,\footcite{fakeTrafficPayday} wird diese Praxis nicht
  selten auch in der Realität umgesetzt.
  
  \item[Rating-Bots]
  Eine weitere Folge des steigenden Konsums im Internet ist die Wichtigkeit von
  positiven Bewertungen. Seien es Güter oder Dienstleistungen, die online
  angeboten werden: eine bedeutende Anzahl an 5-von-5-Sterne-Bewertungen
  verwandelt potenzielle in zahlende Kunden wie kaum ein anderes
  Gütesiegel.\footcite{consumersInfluencedReviews}
  
  \item[DDoS-Bots]
  Die Überlastung eines oder mehrerer Server durch ein Botnetz wird als
  \emph{Distributed Denial of Service}, kurz DDoS, bezeichnet. Die
  Software-Agenten, die für die Überlastung der Server-Infrastruktur
  verantwortlich sind, müssen dabei nicht zwangsweise auf den Rechnern der
  Angreifer laufen. Nicht selten werden diese Bots in Form von Schadware auf
  persönlichen Computern installiert und beteiligen sich an den Angriffen ohne
  die Kenntnisnahme des Besitzers.
\end{description}
