\subsection{Auswertung der Great-Attractor-Daten}
\label{sub:auswertung-der-great-attractor-daten}

\subsubsection{Aufbau des Honeypots}
\label{ssub:aufbau-des-honeypots}

Die Great-Attractor Software wurde im Rahmen dieser Arbeit unter der Domäne
\texttt{motionfill.com} installiert. Als Wirt-Webseite diente ein frisch
installiertes WordPress-CMS in der Version 4.5.2. Die Domäne
\texttt{motionfill.com} wurde verwendet, da sie zum Zeitpunkt der
GA-Installation bereits vier Jahre und sechs Monate alt war. Damit stand die
Chance gut, dass die Domäne bereits in einer Liste von Domänen aufgeführt ist,
die bösartige Software-Agenten abarbeiten würden. Weiters war die Domäne
bereits seit zwei Jahren inaktiv. Die Wahrscheinlichkeit, dass ein
menschlicher Besucher die Domäne aufsuchen würde, war dementsprechend gering.

Am 23. Mai 2016 um 21:06 Uhr (CET) wurde, kurz nach der WordPress-Installation,
der Link \texttt{http://motionfill.com/?r=tw} auf dem sozialen Netzwerk
Twitter\footnote{Permalink zum Tweet:
\url{https://twitter.com/SoftwareAgenten/status/734824609590202368}}
veröffentlicht. Der verwendete Twitter-Account hatte zu diesem Zeitpunk keine
\emph{Twitter-Follower} und war damit praktisch nur für Bots erreichbar. Um
21:19 Uhr wurde die HTTP-404-Fehlerseite von \texttt{https://blogfill.de} so
umgestellt, dass sie alle Anfragen auf \texttt{http://motionfill.com/?r=bf}
umleitete.

Folglich wurden eine Reihe an Einstellungen am WordPress-CMS vorgenommen, um
die Sicherheit des Systems bewusst zu verringern. Das sollte Bots aggressivere
Attacken erlauben.\footnote{Selbstverständlich wurde bei der Installation ein
besonders sicheres Password für den Administratoren-Account gewählt.}

\begin{table}[h]
  \begin{tabularx}{\textwidth}{ l|l|X }
    \textbf{Uhrzeit} & \textbf{Einstellungen} & \textbf{Aktion}\\
    \hline
    21:25 & General Settings \to Membership & \emph{Anyone can register} auf \emph{wahr} setzen \\
    \hline
    21:27 & General Settings \to New User Default Role & auf \emph{Contributor} setzen \\
    \hline
    21:30 & Discussion Settings \to Other comment settings & \emph{Comment author must fill out name and email} auf \emph{falsch} setzen \\
    \hline
    21:31 & Discussion Settings \to Before a comment appears & \emph{Comment author must have a previously approved comment} auf \emph{falsch} setzen \\    
    \hline
    21:33 & Discussion Settings \to Comment Moderation & \emph{Hold a comment in the queue if it contains ??? or more links.} auf \emph{256} setzen \\
  \end{tabularx}
  
  \caption{Die Einstellungen, die am WordPess-CMS vorgenommen wurden}
  \label{tab:wprdPressConfig}
\end{table}

Zudem wurde um 21:36 Uhr ein Artikel aus reinem Blindtext\footnote{TK
\url{https://de.wikipedia.org/wiki/Blindtext} 2016-06-06}
veröffentlicht,\footnote{Adresse:
\texttt{http://motionfill.com/2016/05/23/catch-all/}} der keine weitere
Funktion erfüllen sollte, als für künstliche Aktivitäten zu sorgen, die für
Bots anziehend wirken könnte.

Am 31. Mai 2016 wurde das Experiment um 19:22 Uhr beendet. In dieser Zeitspanne
sammelte das Great-Attractor-System 7.482 einzelne Zugriffe. 3.601 dieser
Zugriffe enthielten HTTP-POST-Daten, mit welchen sich die Bots\footnote{Dass es
sich in diesem Fall um Bots handelt, lässt darauf schließen, dass die meisten
Anmeldeanfragen direkt an den Server gesendet wurden, also ohne zuvor die
Webseite oder die Anmeldeseite zu besuchen. Das legt nahe, dass Bots die Domäne
als aktiv eingestuft hatten und daraufhin direkte HTTP-POST-Anfragen an den
Server sendeten, ein Verhalten, das in in einem handelsüblichen Webbrowser wie
ihn Menschen benutzen nicht möglich ist.} versuchten an der Webseite anzumelden.

\subsubsection{Analyse der gewonnenen Daten}
\label{ssub:analyse-der-gewonnenen-daten}

\begin{table}[h]
  \begin{tabular}{ l|r }
    \textbf{Passwort} & \textbf{Anzahl}\\
    \hline
    ginger & 10 \\
    angela & 10 \\
    adidas & 10 \\
    babydoll & 10 \\
    darkness & 10 \\
    elizabeth & 10 \\
    87654321 & 10 \\
    1234554321 & 9 \\
    natalia & 9 \\
    motorola & 9 \\
    1w2q3r4e & 9 \\
    gold & 9 \\
    555555 & 9 \\
    123123 & 9 \\
    david & 9 \\
    alex & 9 \\
    123465 & 9 \\
    monkey1 & 9 \\
    fatima & 9 \\
    catherine & 9 \\
    hunter & 9 \\
    123zxc & 9 \\
    blahblah & 9 \\
    ricardo & 9 \\
    a1s2d3 & 9 \\
    123qwe & 9 \\
    123654 & 9 \\
    oooooo & 9 \\
    ghbdtn & 9 \\
    sarah & 9 \\
  \end{tabular}
  
  \caption{Die 30 am häufigsten probierten Passwörter}
  \label{tab:maxPasswords}
\end{table}

Ein Großteil der Passwörter in Tabelle \ref{tab:maxPasswords} machen kurze
englische Wörter und einfache Muster einer QWERTY-Tastatur\footnote{Das
QWERTY-Tastaturlayout ist im englischen Sprachraum das am meisten verbreitete
Tastaturlayout und ähnelt anderen Layouts wie dem deutschen QWERTZ- oder dem
französischen AZERTY-Layout.} aus.

\begin{table}[h]
  \begin{tabular}{ l|l|r }
    \textbf{Benutzername} & \textbf{Passwort} & \textbf{Anzahl}\\
    \hline
    admin & 1234567 & 2 \\
    admin & pass & 2 \\
    admin & 123456789 & 2 \\
    admin & 12345678 & 2 \\
    admin & password & 2 \\
    admin & 123123 & 2 \\
    admin & 1234 & 2 \\
  \end{tabular}
  
  \caption{Die meist verwendeten Benutzername-Passwort-Kombinationen}
  \label{tab:maxUserPass}
\end{table}

Die Tabelle \ref{tab:maxUserPass} zeigt alle Anmeldeversuche, die zwei mal
durchgeführt wurden. Alle anderen Kombinationen von Benutzername und Passwort
wurden nur einmal probiert. Angesichts der Tatsache, dass
von den 3.601 Anmeldeversuchen 3.587 von einer einzigen IP-Adresse ausgingen,
die sich auf Russland zurückverfolgen lässt, ergibt das viel Sinn.

\begin{table}[h]
  \begin{tabular}{ l|r }
    \textbf{Benutzername} & \textbf{Anzahl}\\
    \hline
    admin & 334 \\
    support & 334 \\
    tester & 334 \\
    test & 332 \\
    \{host\} & 331 \\
    \{hostname\} & 328 \\
    adm & 324 \\
    administrator & 322 \\
    user2 & 322 \\
    Administrator & 319 \\
    user & 319 \\
    kristleguiffretgo & 1 \\
    florian & 1 \\
  \end{tabular}
  
  \caption{Die am häufigsten probierten Benutzernamen}
  \label{tab:maxUsername}
\end{table}

Im Vergleich zu der Anzahl aller unterschiedlichen probierten Passwörter
(insgesamt 616), wirkt die Anzahl der unterschiedlichen probierten
Benutzernamen (13) sehr klein. Tabelle \ref{tab:maxUsername} zeigt dabei die
vollständige Liste an Benutzernamen samt Anzahl an. Bemerkenswert ist hierbei
der einmalig probierte Benutzername \emph{florian}, der in Kombination mit dem
Passwort \emph{florian} an das Login-System von WordPress gesendet wurde.
Interessant daran ist, dass der tatsächliche Benutzername für die
WordPress-Installation \emph{florian} war, wenn auch mit einem anderen
Passwort. \emph{Florian} ist zudem der Vorname des Besitzers der Domäne
\texttt{motionfill.com}. Wie der Bot, der seine Anfrage aus dem
chinesischen Festland versendet hatte, genau auf diese Information gestoßen ist,
bleibt unklar. Der Name \emph{Florian} taucht beispielsweise nicht in den
WHOIS-Informationen der Domäne \texttt{motionfill.com} auf.

\begin{table}[h]
  \begin{tabularx}{\textwidth}{ X|r }
    \textbf{User-Agent} & \textbf{\#}\\
    \hline
    – & 3624 \\
    jetmon/1.0 (Jetpack Site Uptime Monitor by WordPress.com) & 2276 \\
    Mozilla/5.0 (compatible; MJ12bot/v1.4.5; http://www.majestic12.co.uk/bot.php?+) & 588 \\
    Mozilla/5.0 (compatible; bingbot/2.0; +http://www.bing.com/bingbot.htm) & 116 \\
    Mozilla/5.0 (compatible; Googlebot/2.1; +http://www.google.com/bot.html) & 110 \\
    Java/1.4.1\_04 & 106 \\
    Mozilla/5.0 (Windows NT 6.0; rv:16.0) Firefox/12.0 & 74 \\
    Mozilla/5.0 (compatible; Baiduspider/2.0; +http://www.baidu.com/search/spider.html) & 59 \\
    Mozilla/5.0 (Windows NT 6.1; rv:34.0) Gecko/20100101 Firefox/34.0 & 48 \\
    Java/1.8.0\_91 & 36 \\
    Mozilla/5.0 (Windows NT 6.0; rv:14.0) Gecko/20100101 Firefox/14.0.1 & 30 \\
    Mozilla/5.0 (Windows NT 6.2; WOW64) & 26 \\
  \end{tabularx}
  
  \caption{Die zwölf am meisten verwendeten User-Agents}
  \label{tab:maxUseragent}
\end{table}

Wie in Tabelle \ref{tab:maxUseragent} erkenntlich, war für die Mehrheit der
Anfragen kein User-Agent-HTTP-Header gesetzt. Auf Platz zwei findet sich der
WordPress-eigene Laufzeit-Monitoring-Dienst \emph{Jetpack}. In der Tabelle
finden sich zudem zwei Suchmaschinen-User-Agenten: \texttt{bingbot/2.0} und
\texttt{Googlebot/2.1}. Diese Daten sind mit Vorsicht zu genießen, jeder Bot
kann den eigenen User-Agent frei wählen und daher auch den anderer Bots angeben.

Der vollständige Datenbestand lässt sich online unter
\url{https://github.com/SoftwareAgenten/GA-Archive/tree/master/Great-Attractor}
einsehen. Diese Daten wurden mit dem
\emph{GA-Analyzer}\footnote{\emph{GA-Analyzer} Projektseite:
\url{https://github.com/SoftwareAgenten/GA-Analyzer}} ausgewertet, eine
Spezialsoftware, die eigens für das Untersuchen von
Great-Attractor-Datenbeständen entwickelt wurde.
