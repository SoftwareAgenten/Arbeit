\subsection{Rahmenbedingungen}
\label{sub:forschung_rahmenbedingungen}

Um die Verwendung der \emph{Software-Agenten im Internet}-Softwares zu
erleichtern, wurden sie in Programmiersprachen verfasst, die nicht an eine
Plattform gebunden sind.

\begin{itemize}
\item
  \emph{Great-Attractor} wurde in PHP verfasst, eine verhältnismäßig alte
  Programmiersprache, die nahezu von jedem Webhosting-Dienst unterstützt wird.
  PHP wird für Microsoft Windows, OS~X, Linux und andere Betriebsysteme
  angeboten.
\item
  \emph{GA-Analyzer} und WWWEB wurden für die \emph{Node.js}-Umgebung
  programmiert und lassen sich so ebenfalls unter Microsoft Windows, OS~X und
  Linux ausführen. Während \emph{Great-Attractor} in den meisten Fällen auf
  einem externen Hoster installiert wird, können diese Programme nur auf dem
  eigenen Computer ausgeführt werden. Daher wird \emph{Node.js} in der derzeit
  aktuellsten Version, 6, vorausgesetzt.
\item
  \emph{process\_mailbox}, eine Spezialsoftware, die IMAP-Postfächer in das
  JSON-Datei-Format konvertiert, wurde in der Programmiersprache Go
  geschrieben. Go ist eine kompilierbare Sprache, was die Ausführung deutlich
  beschleunigt. Go ist für Microsoft Windows, OS~X, Linux und andere
  Betriebsysteme verfügbar.
\end{itemize}

Des Weiteren wurden alle gewonnen Daten in für Computer einfach zu parsende
Formate geschrieben und online in
\emph{GA-Archive}\footnote{\url{https://github.com/SoftwareAgenten/GA-Archive}}
veröffentlicht. Zu diesen Formaten zählen JSON, TOML, XML und CSV. Das erlaubt
es anderen Programmen, diese Daten zu importieren und zu verarbeiten.
