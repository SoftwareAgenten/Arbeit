\renewcommand{\abstractname}{Abstract}

\begin{abstract}
  Menschen sind schon lange nicht mehr die einzigen Nutzer des Internets.
  Beinahe 50\,\% aller Webseiten-Aufrufe werden von autonomer Software
  getätigt. Diese auch als Bots bezeichnete Softwares agieren in den Schatten
  des Netzes. Unbemerkt indexieren sie Webseiten, verbreiten Spam, legen
  gefälscht Profile an oder versuchen in Datenbanken einzubrechen.
  
  Hindernisse wie CAPTCHAs oder Honeypots galten bislang als effektive
  Gegenmittel, allerdings verhilft der rapide Fortschritt im Feld der
  Künstlichen Intelligenz modernen Bots auch derartige Barrieren zu
  durchbrechen. Im Bereich E-Mail-Spam findet ein unablässiger Kampf zwischen
  Spam-Bots und Klassifizierungsalgorithmen statt.
  
  Im ersten Teil dieser Arbeit wird das Verhalten und Vermögen von
  Software-Agenten untersucht. Schwerpunkte bilden dabei die Themen Spam und
  Künstliche Intelligenz. Der zweite Teil beschreibt die Anwendung des
  gewonnenen Wissens in Form der Entwicklung eines eigenen Bots, der autonom
  durch das World Wide Web navigiert.
\end{abstract}
