\subsection{Spam} % (fold)
\label{sub:spam}
\subsubsection{Definition} % (fold)
\label{ssub:definition}
Unerwünschte Nachrichten im Internet, größtenteils Werbeangebote, Phishing-Attacken oder Übermittler von Schadsoftware, werden als Spam bezeichnet. Spam-Nachrichten stellen per definitionem eine leidige Kommunikationsform dar. Die alltäglichste Art, namhaft durch ihre Omnipräsenz im Leben mit dem Internet, stellt der E-Mail-Spam dar.

Der Begriff Spam entspringt dem gleichnamigen Dosenfleisch, welches im Zweiten Weltkrieg in großen Mengen an Soldaten verteilt wurde. Der Markenname SPAM (kurz für \emph{Spiced Ham}) setzte sich auf diese Weise rasch im Vereinigten Königreich als Deonym für Frühstücksfleisch durch. Die britische Comedy-Gruppe Monty Python griff im Sketch \emph{Spam}\footnote{\url{https://www.youtube.com/watch?v=M_eYSuPKP3Y}} der BBC Serie \emph{Monty Python’s Flying Circus} die Allgegenwärtigkeit des Fleischprodukts auf, weshalb dieses bis heute als Symbol für einen unerwünschten Überschuss fungiert.
% subsubsection definition (end)

\subsubsection{Spam-Bekämpfung} % (fold)
\label{ssub:spam_bekampfung}
Die Aufnahme von computergeneriertem Spam kann nicht restlos durch eine Blockade der Distribution, beispielsweise durch CAPTCHAs oder Sicherheitsfragen, abgewehrt werden. So können beispielsweise E-Mails versendet werden, ohne dass dies unter der Obrigkeit einer vom Empfänger vertrauten Instanz geschieht. Dieser Typ Spam muss auf Seiten des Adressaten identifiziert und beseitigt werden. Für jene Aufgabe eignen sich im Besonderen Klassifizierungsalgorithmen, welche qua Positiv- und Negativ-Beispielen erlernen Spam-Nachrichten ausfindig zu machen.

Klassifizierungsalgorithmen teilen genannte Beispiele in einen mehrdimensionalen Raum ein. Einzelne Aspekte, die sich von Nachricht zu Nachricht unterscheiden, beschreiben die verschiedenen Dimensionen des Raums. Neue Nachrichten werden ebenfalls in diesem Raum erfasst und mittels Clusteranalyse zugeordnet. Beschriebene Methoden sind den Feldern des Maschinellen Lernens und des Data-Mining angehörig.
% subsubsection spam_bekampfung (end)
% subsection spam (end)
