\subsection{Spam} % (fold)
\label{sub:spam}
\subsubsection{Definition} % (fold)
\label{ssub:definition}
Unerwünschte Nachrichten im Internet, größtenteils Werbeangebote, Phishing-Attacken oder Übermittler von Schadsoftware, werden als Spam bezeichnet. Spam-Nachrichten stellen per definitionem eine leidige Kommunikationsform dar. Die alltäglichste Art, namhaft durch ihre Omnipräsenz im Leben mit dem Internet, stellt der E-Mail-Spam dar.

Die Aufnahme von computergenerierten Spam kann nicht restlos durch eine Blockade der Distribution, beispielsweise durch CAPTCHAs oder Sicherheitsfragen, abgewehrt werden. So können beispielsweise E-Mails versendet werden, ohne das dies unter der Obrigkeit einer vom Empfänger vertrauten Instanz geschieht. Dieser Typ Spam muss auf Seiten des Adressaten identifiziert und beseitigt werden. Für jene Aufgabe eignen sich im Besonderen Klassifizierungsalgorithmen, welche qua Positiv- und Negativ-Beispielen erlernen Spam-Nachrichten ausfindig zu machen.

Klassifizierungsalgorithmen teilen genannte Beispiele in einen mehrdimensionalen Raum ein. Einzelne Aspekte, die sich von Nachricht zu Nachricht unterscheiden, beschreiben die verschiedenen Dimensionen des Raums. Neue Nachrichten werden ebenfalls in diesem Raum erfasst und mittels Clusteranalyse zugeordnet. Beschriebene Methode sind den Feldern des Maschinellem Lernen und des Data-Mining angehörig.
% subsubsection definition (end)
% subsection spam (end)